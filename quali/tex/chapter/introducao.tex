Two main types of optimal covering problems can be found in the literature: the Minimum Cover Problem, also known as just Set Cover Problem, and the Maximal Covering Problem \cite{karatas}. 

One of the 21 Karp's NP-Complete problems\footnote{The decision version, which asks if there is a cover of size $k$, is NP-Complete.} \cite{karp}, the Minimum Cover Problem is very well explored and considered to be a classic. 
Given a demand set and a collection of subsets of the demand set, the problem asks what is the minimum number of elements from the collection of subsets needed to cover the whole demand set. One of its most famous examples is the Minimum Vertex Cover defined over graphs, where the vertex set has to be covered by a subset of edges.

The second type of covering problems arose from the fact that covering almost all the demand set can be a lot cheaper than having to cover it all \cite{garcia}. This second type is known as \sigla{\MCLP}{Maximal Covering Location Problem} and was introduced in \cite{church:1974}. In this first study, it is defined on a network with demand nodes, a facility set is also given and a solution maximizes the demand coverage satisfying the constraint that only a subset of the facilities is used. Just like the Minimum Cover, MCLP is a NP-Hard problem \cite{hatta:2013} and both deterministic, using integer programming \cite{church:1974}, and heuristic methods \cite{revelle:2008} have been proposed to solve it.

In \cite{church:1984} a new kind of MCLP named \sigla{\PMCLP}{Planar Maximal Covering Location Problem} was introduced. This version of the problem was not defined on a network, instead the demand set and the facilities are located in $\R^2$ having the coverage area of a facility be defined by a distance function. PMCLP is said to have been studied under Euclidean and rectilinear distance functions \cite{younies}. The Euclidean norm PMCLP, which has a lot of results that can be applied for the elliptical PMCLP, is also found in the literature as the problem of maximization of points covered by a fixed number of unit disks \cite{cabello:2006}. 
Early works only tackled the one-disk version of the problem, in \cite{chazelle:1986} a $\bigO(n^2)$ algorithm, which still stands as the best in terms of run-time complexity, was proposed beating the prior $\bigO(n^2\log{n})$ algorithm created by \cite{drezner}.
The $m$ unit disks maximal covering was studied in \cite{cabello:2006} which had as its most important result a $(1-\epsilon)$-approximation algorithm which runs in $\bigO(n\log{n})$. To achieve its main goal, however, they developed a deterministic $\bigO(n^{2m-1}\log{n})$ algorithm which gets employed into their approximation scheme.
Additionally, in \cite{aronov:2008} one-disk maximal covering is proven to be 3SUM-HARD. This means that maximizing the amount of points covered by a disk is as hard as finding three real numbers that sum to zero among $n$ given real numbers.

Planar maximal covering with ellipses differs from its disks counterpart only in the shape of the facility's coverage area. The main motivation to study this modified version is that cellphone towers can have elliptical shaped coverage area, so in order to determine what are the best locations to place $m$ cellphone towers to maximize the amount of the population covered by its signal, an elliptical PMCLP is better suited \cite{canbolat}. Only two articles have been found published in the literature that study this problem. In \cite{canbolat}, a mixed-integer non-linear programming method was proposed as a first approach to the problem. For some instances the method took too long and did not find an optimal solution. For this reason a heuristic method was developed using a technique called Simulated Annealing. Solutions for the instances that timed-out with the first method were then obtained. The problem was further explored in \cite{andreta} which proposed a deterministic method that showed better performance obtaining optimal solutions for the instances which the first method could not. Also, in \cite{andreta}, a version of the problem where every ellipse can be freely rotated was introduced and an exact method, which could not find optimal solutions for large instances, and a heuristic method were proposed for it. Despite the similarities, none of the works cited above base their development on the maximal covering with disks algorithms found in the literature.


This work is structured in the following way: \autoref{chapter:definitions} introduces some definitions and results that are used throughout the next chapters; in \autoref{chapter:pmclp}, the maximal covering by disks problem is studied and a $\bigO(n^{2m})$ algorithm is proposed; in \autoref{chapter:ellipses}, the maximal covering by ellipses is introduced and the algorithm for the disks case is adapted for it; finally, \autoref{chapter:future_work} presents what is left as future work. Also, \autoref{chapter:ellipses_intersection} determines with detail the intersection of two ellipses, which is used in the algorithm developed in \autoref{chapter:ellipses}.

