\documentclass{beamer}

\mode<presentation>{
\usetheme{Madrid}
%\usecolortheme{beaver}
}
\usepackage[utf8]{inputenc}
%\usepackage{default}
%\usepackage[portuguese]{babel}
%\usepackage{pgfplots}
%\pgfplotsset{/pgf/number format/use comma,compat=newest}
%\usepackage{color}
\usepackage{amsfonts}
\usepackage{mathrsfs}  

%\usepackage{hyperref}
%\usepackage{tikz}

%MEUS COMANDOs
\newcommand{\R}{\mathbb{R}}
\newcommand{\D}{\mathscr{D}}
\newcommand{\Pp}{\mathscr{P}}
\newcommand{\Cc}{\mathscr{C}}
\newcommand{\E}{\mathscr{E}}



\newcommand{\bigO}{\mathscr{O}}

\title[Qualificação de Mestrado]{Planar Maximal Covering with Ellipses}
\author[Tedeschi, D. F.]{Danilo F. Tedeschi}
\institute[ICMC]{Instituto de Ciências Matemáticas e Computação}
\date{\today}

\begin{document}

\begin{frame}
 \maketitle
\end{frame}

\begin{frame}
\frametitle{Sumário}
 \tableofcontents
\end{frame}

\section{Introdução}
\begin{frame}
\frametitle{Introdução}
\begin{itemize}
	\item Problemas de cobertura
	\begin{itemize}
		%\item Cobertura mínima
		\item Cobertura maximal
	\end{itemize}
	\item Maximal Covering Location Problem (MCLP)
	\item Planar Maximal Covering Location Problem (PMCLP)
	\begin{itemize}
	\item Um disco: algoritmos $\bigO(n^2)$ e $\bigO(n^2\log{n})$
	\item $m$ discos: algoritmo $\bigO(n^{2m-1}\log{n})$
	\end{itemize}
	\item Objetivos
	\begin{itemize}
		\item Desenvolver algoritmo $\bigO(n^2\log{n})$ para um disco
		\item Adaptar o algoritmo para o caso com $m$ ellipses
	\end{itemize}
\end{itemize}

\end{frame} 


\section{Preliminares}

\begin{frame}
	\frametitle{Preliminares}
	\begin{block}{Norma Euclidiana}
		Seja $u \in \R^2$
		
		\begin{equation*}
		||u||_2 = \sqrt{u^Tu}
		\end{equation*}
		
	\end{block}

\begin{block}{Norma Elíptica}
	Seja $u \in \R^2$ e $Q$ uma matriz $2x2$ positiva definida
	
	\begin{equation*}
	||u||_{Q} = \sqrt{u^TQu}
	\end{equation*}
\end{block}
\end{frame}

\begin{frame}{Preliminares}
	\begin{block}{Disco}
		content...
	\end{block}

	\begin{block}{Ellipse}
		content...
	\end{block}
\end{frame}

\section{Cobertura Maximal por Discos}
\begin{frame}{Cobertura maximal por discos}{Um disco}
	content... trabalhos passados e motivacao
\end{frame}

\section{Cobertura Maximal por Discos}
\begin{frame}{Cobertura maximal por discos}{Um disco}
	content... meu algoritmo
\end{frame}


\section{Cobertura Maximal por Ellipses}

\begin{frame}{Cobertura Maximal por Ellipses}
	content... trabalhos passados
\end{frame}

\begin{frame}{Cobertura Maximal por Ellipses}{$m$ elipses}
	content... algoritmo
\end{frame}

\section{Trabalhos Futuros}

\begin{frame}{Trabalhos Futuros}

\end{frame}
\end{document}