\documentclass{article}

\usepackage{amsfonts}
\usepackage{amsmath}
\usepackage{mathrsfs}

\usepackage{tikz}
\usetikzlibrary{quotes, angles, intersections}
\newcommand{\R}{\mathbb{R}}
\newcommand{\D}{\mathscr{D}}
\newcommand{\Pp}{\mathscr{P}}
\newcommand{\Cc}{\mathscr{C}}
\newcommand{\E}{\mathscr{E}}

\newcommand{\bigO}{\mathscr{O}}
%\usetikzlibrary{matrix}
%\usepackage{pgfplots}
%\pgfplotsset{compat=1.4}
\usepackage{algorithm2e}
\usepackage{algorithmic}

\usepackage{caption}
\newcommand{\source}[1]{\caption*{Source: {#1}} }
\newcommand{\detE}[6]{\dfrac{(#5-#1)(#5 - #3) + q(#6-#2)(#6-#4)}{(#6-#2)(#5-#3)-(#6-#4)(#5-#1)}}
\begin{document}
\section{Intro}

	Points:
\begin{equation}
\begin{split}
	p_i = (x_i, y_i)\\
	p_1 = (0,0)
\end{split}
\end{equation}

If $p_i$ are not co-linear and the ellipse with form:

\begin{equation}
x^2 + qy^2 = a^2b^2
\end{equation}

can be determined by three points:

\begin{equation}\label{eq:eq1}
\detE{x_1}{y_1}{x_2}{y_2}{x}{y} = 	\detE{x_1}{y_1}{x_2}{y_2}{x_3}{y_3}
\end{equation}

$q$ is fixed, we want to find $\theta$, let

\begin{equation}
M(\theta) = \left[\begin{array}{cc} \cos{\theta} & -\sin{\theta}\\ \sin{\theta} & \cos{\theta}\end{array}\right]
\end{equation}

Now, lets define

\begin{eqnarray}
u_i = M(\theta)(p_i-c)\\
c=(h,k)
\end{eqnarray}

as $u_1=0$, we have

\begin{equation}
\dfrac{X^t(X-u_2)}{det(X, X-u_2)} = \dfrac{u_3^t(u_3-u_2)}{det(u_3, u_3-u_2)}
\end{equation}

\subsection{Right hand side of the equation}

The dot product after, with the rotation applied is:

\begin{align*}
u^tw = \cos^2{\theta}(u_xw_x + qu_yw_y)+\\
\sin{\theta}\cos{\theta}((q-1)(u_yw_x+u_xw_y))+\\
\sin^2{\theta}(u_yw_y + qu_xw_x)
\end{align*}

The determinant, also with the rotation is:

\begin{align*}
det(M(\theta)v, M(\theta)w)=\\
\left|\begin{array}{cc}M(\theta)_1 v & M(\theta)_2v\\ M(\theta)_1 w & M(\theta)_2w\end{array}\right|=\\
|M(\theta)|\left|\begin{array}{cc} v_x & v_y \\ w_x & w_y \end{array}\right|=\left|\begin{array}{cc} v_x & v_y \\ w_x & w_y \end{array}\right|=v_xw_y - v_yw_x
\end{align*}

Now, grouping the constants:

\begin{align*}
\Gamma_1 = \frac{p_{3x}(p_{3x}-p_{2x}) + qp_{3y}(p_{3y}-p_{2y})}{det(p_3, p_3-p_2)}\\
\Gamma_2 = \frac{(q-1)(p_{3y}(p_{3x}-p_{2x})+p_{3x}(p_{3y}-p_{2y}))}{det(p_3, p_3-p_2)}\\
\Gamma_3 = \frac{p_{3y}(p_{3y}-p_{2y}) + qp_{3x}(p_{3x}-p_{2x})}{det(p_3,p_3-p_2)}
\end{align*}

Getting everything together we get

\newcommand{\rhse}{\Gamma_1\cos^2{\theta} + \Gamma_2\cos{\theta}\sin{\theta} + \Gamma_3\sin^2{\theta}}

\begin{align}
\dfrac{u_3^t(u_3-u_2)}{det(p_3, p_3-p_2)}=\rhse
\end{align}

\subsection{The left rand side of the equation}
Looking at the first part:

\begin{align*}
 	(X-u_1)^t(X-u_2)=X^t(X-u_2)=X^tX -X^tM(\theta)p_2=\\
 	x^2 + qy^2 - x((p_{2x}-h)\cos{\theta} - (p_{2y}-k)\sin{\theta}) - qy((p_{2x}-h)\sin{\theta} + (p_{2y}-k)\cos{\theta})
\end{align*}

and

\begin{align*}
det(X,X-u_2)=\left|
\begin{array}{cc}
x & y\\
x - M(\theta)_1p_2 & y - M(\theta)_2p_2 
\end{array}
\right| =\\
\left|
\begin{array}{cc}
	x & y\\
	x -(p_{2x}\cos{\theta} - p_{2y}\sin{\theta}) & y - (p_{2x}\sin{\theta} + p_{2y}\cos{\theta}) 
\end{array}
\right|=\\
y(p_{2x}\cos{\theta} - p_{2y}\sin{\theta}) - x(p_{2x}\sin{\theta} + p_{2y}\cos{\theta})
\end{align*}

Making:

\begin{eqnarray}
\gamma = p_{2x}\cos{\theta} - p_{2y}\sin{\theta}\\
\delta = p_{2x}\sin{\theta} + p_{2y}\cos{\theta}\\
\Phi = \rhse
\end{eqnarray}

Then, we get:

\begin{equation}
\dfrac{(X-u_1)^t(X-u_2)}{det(X-u_1, X-u_2)} = \dfrac{x^2 + qy^2 - x\gamma -qy\delta}{y\gamma - x\delta} = \Phi
\end{equation}

\subsection{Doing all the maths}

Now, we need to rearrange everything so it would look like an ellipse equation.

\begin{eqnarray*}
x^2 + qy^2 - x(\gamma -\Phi\delta) - qy(\delta + \frac{\Phi\gamma}{q}) = 0\\
(x - \frac{\gamma -\Phi\delta}{2})^2 + q(y - \frac{q\delta + \Phi\gamma}{2q})^2 = R\\
R= \frac{(\gamma - \Phi\delta)^2}{4} + \frac{(q\delta + \Phi\gamma)^2}{4q^2}
\end{eqnarray*}

As $q$ is a fixed parameter and we are rotating the points, we need to impose that $R=a^2$, so it satisfies the equation we started for an ellipse.

\section{Having $R=a^2$, it looks way easier than it is...}


Let us do it:


$$ \frac{(\gamma - \Phi\delta)^2}{4} + \frac{(q\delta + \Phi\gamma)^2}{4q^2} = a^2$$
$$ q^2(\gamma^2 - 2\Phi\gamma\delta + \Phi^2\delta^2) + q^2\delta^2 + 2q\delta \Phi \gamma + \Phi^2\gamma^2 = 4q^2a^2= $$

$$
q^2\gamma^2 + q^2\Phi^2\delta^2 + q^2\delta^2 +\Phi^2\gamma^2  + 2\Phi\gamma\delta(q - q^2)= 4q^2a^2
$$

$$
\gamma^2(q^2 +\Phi^2) + \delta^2[q^2(1+\Phi^2)] + \gamma\delta[2\Phi(q-q^2)]= 4q^2a^2
$$

Then,
\newcommand{\ct}{\cos{\theta}}
\newcommand{\ctt}[1]{\cos^{#1}{\theta}}
\newcommand{\st}{\sin{\theta}}
\newcommand{\stt}[1]{\sin^{#1}{\theta}}


\begin{eqnarray}
\gamma^2 = p_{2x}^2\cos^2{\theta} - p_{2x}p_{2y}\sin{2\theta} + p_{2y}^2\sin^2{\theta}\\
\delta^2 = p_{2x}^2\sin^2{\theta} + p_{2x}p_{2y}\sin{2\theta}+ p_{2y}^2\cos^2{\theta}\\
\delta\gamma = \cos^2{\theta}(p_{2x}p_{2y}) + \frac{\sin{(2\theta)}}{2}(p_{2x}^2 - p_{2y}^2) - \sin^2{\theta}(p_{2x}p_{2y})\\
\Phi^2=\Gamma_1^2\ctt{4} + 2\Gamma_1\Gamma_2\ctt{3}\st + (2\Gamma_1\Gamma_3 + \Gamma_2^2)\ctt{2}\stt{2} + 2\Gamma_2\Gamma_3\ct\stt{3} + \Gamma_3^2\stt{4}
\end{eqnarray}

We have

\begin{eqnarray*}
\cos^2{\theta}[(q^2+\Phi^2)p_{2x}^2 + q^2(\Phi^2+1)p_{2y}^2 + 2\Phi(q-q^2)p_{2x}p_{2y}]+\\
\sin^2{\theta}[(q^2+\Phi^2)p_{2y}^2 + q^2(\Phi^2+1)p_{2x}^2 - 2\Phi(q-q^2)p_{2x}p_{2y}]+\\
\sin{(2\theta)}[\Phi(q-q^2)(p_{2x}^2-p_{2y}^2) - (q^2+\Phi^2)p_{2x}p_{2y} + q^2(\Phi^2+1)p_{2x}p_{2y}]=4q^2a^2
\end{eqnarray*}



Let's call sines and cosines $s$ and $r$, with the condition that $$s^2+r^2-1=0$$

Renaming the constants on $\Phi^2$:

$$\Phi^2 = \Phi_1 r^4 + \Phi_2 r^3s + \Phi_3r^2s^2 + \Phi_4rs^3 + \Phi_5s^4$$

$$\Phi=\Gamma_1r^2 + \Gamma_2rs + \Gamma_3s^2$$

Now let's work on each of the terms separately:
\subsection{First one $\cos^2(\theta)$}
\begin{align*}
r^2(\Phi^2(p_{2x}^2 + q^2p_{2y}^2) + 2\Phi(q-q^2)p_{2x}p_{2y} + q^2(p_{2x}^2+p_{2y}^2))=\\
(p_{2x}^2+q^2p_{2y}^2)(\Phi_1 r^6 + \Phi_2 r^5s + \Phi_3r^4s^2 + \Phi_4r^3s^3 + \Phi_5r^2s^4) +\\
2(q-q^2)p_{2x}p_{2y}(\Gamma_1r^4 + \Gamma_2r^3s + \Gamma_3r^2s^2)+\\
q^2(p_{2x}^2+p_{2y}^2))r^2
\end{align*}

Grouping them

\begin{align*}
r^6[(p_{2x}^2+q^2p_{2y}^2)\Phi_1]+\\
r^5s[\Phi_2(p_{2x}^2+q^2p_{2y}^2)]+\\
r^4s^2[\Phi_3(p_{2x}^2+q^2p_{2y}^2)]+\\
r^4[2(q-q^2)p_{2x}p_{2y}\Gamma_1]+\\
r^3s^3[\Phi_4(p_{2x}^2+q^2p_{2y}^2)]+\\
r^3s[2(q-q^2)p_{2x}p_{2y}\Gamma_2]+\\
r^2s^4[\Phi_5(p_{2x}^2+q^2p_{2y}^2)]+\\
r^2s^2[2(q-q^2)p_{2x}p_{2y}\Gamma_3]+\\
r^2[q^2(p_{2x}^2+p_{2y}^2)]
\end{align*}

\subsection{Second one $\sin^2{\theta}$}

\begin{align*}
s^2{\theta}[(q^2+\Phi^2)p_{2y}^2 + q^2(\Phi^2+1)p_{2x}^2 - 2\Phi(q-q^2)p_{2x}p_{2y}]=\\
s^2(\Phi^2(p{2y}^2+q^2p_{2x}^2)-2\Phi(q-q^2)p_{2x}p_{2y}+q^2(p_{2x}^2+p_{2y}^2))
\end{align*}

Almost the same as the first one:

\begin{align*}
r^6[\Phi_1(p_{2y}^2+q^2p_{2x}^2)]+\\
r^5s[\Phi_2(p_{2y}^2+q^2p_{2x}^2)]+\\
r^4s^2[\Phi_3(p_{2y}^2+q^2p_{2x}^2)]+\\
r^4[-2(q-q^2)p_{2x}p_{2y}\Gamma_1]+\\
r^3s^3[\Phi_4(p_{2y}^2+q^2p_{2x}^2)]+\\
r^3s[-2(q-q^2)p_{2x}p_{2y}\Gamma_2]+\\
r^2s^4[\Phi_5(p_{2y}^2+q^2p_{2x}^2)]+\\
r^2s^2[-2(q-q^2)p_{2x}p_{2y}\Gamma_3]+\\
r^2[q^2(p_{2x}^2+p_{2y}^2)]
\end{align*}

\subsection{The last one $\sin{\theta}\cos{\theta}$}

\begin{align*}
2rs[\Phi(q-q^2)(p_{2x}^2-p_{2y}^2) - (q^2+\Phi^2)p_{2x}p_{2y} + q^2(\Phi^2+1)p_{2x}p_{2y}]=\\
2rs[\Phi^2(p_{2x}p_{2y}(q^2-1)) + \Phi(q-q^2)(p_{2x}^2-p_{2y}^2)]=\\
(2p_{2x}p_{2y}(q^2-1))(\Phi_1 r^5s + \Phi_2 r^4s^2 + \Phi_3r^3s^3 + \Phi_4r^2s^4 + \Phi_5rs^5)+\\
2(q-q^2)(p_{2x}^2-p_{2y}^2)(\Gamma_1r^3s + \Gamma_2r^2s^2 + \Gamma_3rs^3)
\end{align*}
Grouping the monomiums:
\newcommand{\pcoeff}{2p_{2x}p_{2y}(q^2-1)}
\newcommand{\pcoef}{2(q-q^2)(p_{2x}^2-p_{2y}^2)}
\begin{align*}
r^5s[\pcoeff\Phi_1]+\\
r^4s[\pcoeff\Phi_2]+\\
r^3s^3[\pcoeff\Phi_3]+\\
r^3s[\pcoef\Gamma_1]+\\
r^2s^4[\pcoeff\Phi_4]+\\
r^2s^2[\pcoef\Gamma_2]+\\
rs^5[\pcoeff\Phi_5]+\\
rs^3[\pcoef\Gamma_3]
\end{align*}

\subsection{Putting all this mess together once again}
Let's do this

\begin{align*}
r^6[(p_{2x}^2+q^2p_{2y}^2 + p_{2y}^2+q^2p_{2x}^2)\Phi_1]+\\
r^5s[(p_{2x}^2+q^2p_{2y}^2 + p_{2y}^2+q^2p_{2x}^2)\Phi_2 + \pcoeff\Phi_1]+\\
r^4s^2[(p_{2x}^2+q^2p_{2y}^2 + p_{2y}^2+q^2p_{2x}^2)\Phi_3]+\\
r^3s^3[\pcoeff\Phi_3 + (p_{2x}^2+q^2p_{2y}^2 + p_{2y}^2+q^2p_{2x}^2)\Phi_4]+\\
r^3s[\pcoef\Gamma_1]+\\
r^2s^4[(p_{2x}^2+q^2p_{2y}^2 + p_{2y}^2+q^2p_{2x}^2)\Phi_3 + \pcoeff\Phi_4]+\\
r^2s^2[\pcoef\Gamma_2]+\\
r^2[2q^2(p_{2x}^2+p_{2y}^2)]+\\
rs^5[\pcoeff\Phi_5]+\\
rs^3[\pcoef\Gamma_3]=4a^2q^2
\end{align*}

Renaming the constants, we get:

\begin{align*}
A_1r^6 + A_2r^5s + A_3r^4s^2 + A_4r^3s^3 + A_5r^3s + A_6r^2s^4 + A_6r^2s^2 + A_7r^2 + A_8rs^5 + A_9rs^3 - 4a^2q^2 = 0\\
r^2 + s^2 -1 = 0
\end{align*}
\end{document}