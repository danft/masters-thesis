\documentclass{article}

\usepackage{amsfonts}
\usepackage{amsmath}
\usepackage{mathrsfs}

\usepackage{tikz}
\usetikzlibrary{quotes, angles, intersections}
\newcommand{\R}{\mathbb{R}}
\newcommand{\D}{\mathscr{D}}
\newcommand{\Pp}{\mathscr{P}}
\newcommand{\Cc}{\mathscr{C}}
\newcommand{\E}{\mathscr{E}}

\newcommand{\bigO}{\mathscr{O}}
%\usetikzlibrary{matrix}
%\usepackage{pgfplots}
%\pgfplotsset{compat=1.4}
\usepackage{algorithm2e}
\usepackage{algorithmic}

\usepackage{caption}
\newcommand{\source}[1]{\caption*{Source: {#1}} }

\begin{document}
	Points:
\begin{equation}
\begin{split}
	p_i = (x_i, y_i)\\
	p_1 = (0,0)
\end{split}
\end{equation}

If $p_i$ are not co-linear and the ellipse with form:

\begin{equation}
x^2 + qy^2 = a^2b^2
\end{equation}

can be determined by three points:

\begin{equation}\label{eq:eq1}
\dfrac{(x-x_1)(x - x_2) + q(y-y_1)(y-y_2)}{(y-y_1)(x-x_2)-(y-y_2)(x-x_1)} = 	\dfrac{(x_3-x_1)(x_3 - x_2) + q(y_3-y_1)(y_3-y_2)}{(y_3-y_1)(x_3-x_2)-(y_3-y_2)(x_3-x_1)}
\end{equation}

$q$ is fixed, we want to find $\theta$, let

\begin{equation}
M(\theta) = \left[\begin{array}{cc} \cos{\theta} & -\sin{\theta}\\ \sin{\theta} & \cos{\theta}\end{array}\right]
\end{equation}

Now, lets define

\begin{eqnarray}
u_i = M(\theta)(p_i-c)\\
c=(h,k)
\end{eqnarray}

we can rearrange \ref{eq:eq1}

\begin{equation}
\dfrac{(X-u_1)^t(X-u_2)}{det(X-u_1, X-u_2)} = \dfrac{(u_3-u_1)^t(u_3-u_2)}{det(u_3-u_1, u_3-u_2)}
\end{equation}

Analyzing the second part:

\begin{align*}
u^tv = \cos^2{\theta}(u_xv_x + qu_yv_y)+\\
\sin{\theta}\cos{\theta}((q-1)(u_yv_x+u_xv_y))+\\
\sin^2{\theta}(u_yv_y + qu_xv_x)
\end{align*}

Now looking at $det(u_3, u_3-u_2)$

\begin{equation}
\begin{split}
det(u_3, u_3-u_2) = det(M(\theta)v, M(\theta)w)=\\
\left|\begin{array}{cc}M(\theta)_1 v & M(\theta)_2v\\ M(\theta)_1 w & M(\theta)_2w\end{array}\right|=
\end{split}
\end{equation}

\begin{equation*}
\begin{split}
(v_x\cos{\theta}-v_y\sin{\theta})(w_x\sin{\theta} + w_y\cos{\theta})-(\sin{\theta}v_x + \cos{\theta}v_y)(w_x\cos{\theta} - w_y\sin{\theta})=\\
\cos^2{\theta}(v_xw_y-v_yw_x)+\sin^2{\theta}(-v_yw_x+v_xw_y) + \sin{\theta}\cos{\theta}(v_xw_x-v_yw_y-v_xw_x+v_yw_y)=\\
v_xw_y(\cos^2{\theta}+\sin^2{\theta}) - v_yw_x(\cos^2{\theta}+\sin^2{\theta})=\\
v_xw_y-v_yw_x=det(v, w) = det(u_3, u_3-u_2) = det(p_3, p_3-p_2)
\end{split}
\end{equation*}

Then, the second part is:

\begin{equation}
\dfrac{u_3^t(u_3-u_2)}{det(p_3, p_3-p_2)}
\end{equation}

The main equation is:

\begin{equation}
\dfrac{(X-u_1)^t(X-u_2)}{det(X-u_1, X-u_2)} = \dfrac{u_3^t(u_3-u_2)}{det(p_3, p_3-p_2)}
\end{equation}

Looking at the first part:

\begin{eqnarray*}
 	(X-u_1)^t(X-u_2)=X^t(X-u_2)=X^tX -X^tM(\theta)p_2=\\
 	x^2 + qy^2 - x((p_{2x}-h)\cos{\theta} - (p_{2y}-k)\sin{\theta}) - qy((p_{2x}-h)\sin{\theta} + (p_{2y}-k)\cos{\theta})
\end{eqnarray*}

and

\begin{eqnarray*}
det(X,X-u_2)=\left|
\begin{array}{cc}
x & y\\
x - M(\theta)_1p_2 & y - M(\theta)_2p_2 
\end{array}
\right| =\\
\left|
\begin{array}{cc}
	x & y\\
	x -(p_{2x}\cos{\theta} - p_{2y}\sin{\theta}) & y - (p_{2x}\sin{\theta} + p_{2y}\cos{\theta}) 
\end{array}
\right|=\\
y(p_{2x}\cos{\theta} - p_{2y}\sin{\theta}) - x(p_{2x}\sin{\theta} + p_{2y}\cos{\theta})
\end{eqnarray*}

Making:

\begin{eqnarray}
\gamma = p_{2x}\cos{\theta} - p_{2y}\sin{\theta}\\
\gamma = p_{2x}\frac{1-t^2}{1+t^2} - p_{2y}\frac{2t}{1+t^2}\\
\delta = p_{2x}\sin{\theta} + p_{2y}\cos{\theta}\\
\gamma = p_{2x}\frac{2t}{1+t^2} + p_{2y}\frac{1-t^2}{1+t^2}\\
\Phi=\dfrac{u_3^t(u_3-u_2)}{det(p_3, p_3-p_2)} = \Phi_a\cos^2{\theta} + \Phi_b \sin{(2\theta)} + \Phi_c \sin^2{\theta}=\\
\Phi=\Phi_a(\frac{1-t^2}{1+t^2})^2 + \Phi_b(\frac{4t(1-t^2)}{(1+t^2)^2}) + \Phi_c(\frac{2t}{1+t^2})^2\\
\frac{1}{4}((\Phi_a-\Phi_c) \cos{2\theta} + 2\Phi_b\sin{2\theta} + \Phi_a + \Phi_c)=\\
\Phi_a\cos{2\theta} + \Phi_b\sin{2\theta} + \Phi_c
\end{eqnarray}

Then, we get:

\begin{equation}
\dfrac{(X-u_1)^t(X-u_2)}{det(X-u_1, X-u_2)} = \dfrac{x^2 + qy^2 - x\gamma -qy\delta}{y\gamma - x\delta} = \Phi
\end{equation}

Rearranging

\begin{eqnarray*}
x^2 + qy^2 - x(\gamma -\Phi(\delta)) - qy(\delta + \frac{\Phi\gamma}{q}) = 0\\
(x - \frac{\gamma -\Phi\delta}{2})^2 + q(y - \frac{q\delta + \Phi\gamma}{2q})^2 = R\\
R= \frac{(\gamma - \Phi\delta)^2}{4} + \frac{(q\delta + \Phi\gamma)^2}{4q^2}
\end{eqnarray*}


\section{Wrong one}


Then, as $q$ is fixed, we need to impose:
$$ \frac{(\gamma - \Phi\delta)^2}{4} + \frac{(q\delta + \Phi\gamma)^2}{4q^2} = a^2$$
$$ q^2(\gamma^2 - 2\Phi\gamma\delta + \Phi^2\delta^2) + q^2\delta^2 + 2q\delta \Phi \gamma + \Phi^2\gamma^2 = 4q^2a^2= $$

$$
q^2\gamma^2 + q^2\Phi^2\delta^2 + q^2\delta^2 +\Phi^2\gamma^2  + 2\Phi\gamma\delta(q - q^2)= 4q^2a^2
$$

$$
\gamma^2(q^2 +\Phi^2) + \delta^2[q^2(1+\Phi^2)] + \gamma\delta[2\Phi(q-q^2)]= 4q^2a^2
$$

Then,

\begin{eqnarray}
\gamma^2 = p_{2x}^2\cos^2{\theta} - p_{2x}p_{2y}\sin{2\theta} + p_{2y}^2\sin^2{\theta}\\
\delta^2 = p_{2x}^2\sin^2{\theta} + p_{2x}p_{2y}\sin{2\theta}+ p_{2y}^2\cos^2{\theta}\\
\delta\gamma = \cos^2{\theta}(p_{2x}p_{2y}) + \frac{\sin{(2\theta)}}{2}(p_{2x}^2 - p_{2y}^2) - \sin^2{\theta}(p_{2x}p_{2y})\\
\Phi^2=
\end{eqnarray}

We have

\begin{eqnarray*}
\cos^2{\theta}[(q^2+\Phi^2)p_{2x}^2 + q^2(\Phi^2+1)p_{2y}^2 + 2\Phi(q-q^2)p_{2x}p_{2y}]+\\
\sin^2{\theta}[(q^2+\Phi^2)p_{2y}^2 + q^2(\Phi^2+1)p_{2x}^2 - 2\Phi(q-q^2)p_{2x}p_{2y}]+\\
\sin{(2\theta)}[\Phi(q-q^2)(p_{2x}^2-p_{2y}^2) - (q^2+\Phi^2)p_{2x}p_{2y} + q^2(\Phi^2+1)p_{2x}p_{2y}]=4q^2a^2
\end{eqnarray*}

-----

\begin{eqnarray*}
	(\frac{1-t^2}{1+t^2})^2[(q^2+\Phi^2)p_{2x}^2 + q^2(\Phi^2+1)p_{2y}^2 + 2\Phi(q-q^2)p_{2x}p_{2y}]+\\
	(\frac{2t}{1+t^2})^2[(q^2+\Phi^2)p_{2y}^2 + q^2(\Phi^2+1)p_{2x}^2 - 2\Phi(q-q^2)p_{2x}p_{2y}]+\\
		(\frac{4t(1-t^2)}{1+t^2})[\Phi(q-q^2)(p_{2x}^2-p_{2y}^2) - (q^2+\Phi^2)p_{2x}p_{2y} + q^2(\Phi^2+1)p_{2x}p_{2y}]=4q^2a^2
\end{eqnarray*}


Which can be written as

\begin{eqnarray}
A\cos^2{\theta} + B\sin{(2\theta)} + C\sin^2{\theta} = D\\
A,B,C,D > 0
\end{eqnarray}

Which can be rearranged as:

\begin{eqnarray*}
\frac{A-C}{2}\cos{(2\theta)} + B\sin{(2\theta)} + \frac{A+C-2D}{2} = 0
\end{eqnarray*}

Rewritting everything (keep up!):

\begin{equation*}
\begin{split}
\hat{A}=\frac{A-C}{2}\\
\hat{B}=B\\
\hat{C}=\frac{A+C-2D}{2}\\
\hat{A} \cos{(2\theta)} + \hat{B}\sin{(2\theta)} + \hat{C} = 0
\end{split}
\end{equation*}

Then,

\begin{eqnarray}
\hat{A} \cos{(t)} + \hat{B}\sin{(t)} + \hat{C} = \hat{C} + \sqrt{\hat{A}^2 + \hat{B}^2} \sin{(t+\alpha)} = 0\\
t + \alpha = \arcsin{\frac{-\hat{C}}{\sqrt{\hat{A}^2 + \hat{B}^2}}}
\end{eqnarray}

where $$\tan{\alpha} = \frac{\hat{A}}{\hat{B}} \Rightarrow \alpha = \arctan{\frac{\hat{A}}{\hat{B}}}$$

Then, finally

\begin{equation}
	\theta = \arcsin{\frac{-\hat{C}}{2\sqrt{\hat{A}^2 + \hat{B}^2}}} - \frac{1}{2}\arctan{\frac{\hat{A}}{\hat{B}}}
\end{equation}

Similirarly we can achieve the other result:

\begin{equation}
\theta = \arccos{\frac{-\hat{C}}{2\sqrt{\hat{A}^2 + \hat{B}^2}}} + \frac{1}{2}\arctan{\frac{\hat{A}}{\hat{B}}}
\end{equation}
	
	
	%https://brilliant.org/wiki/trigonometric-equations/
\end{document}