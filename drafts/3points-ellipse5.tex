\documentclass{article}

\usepackage{amsfonts}
\usepackage{amsmath}
\usepackage{mathrsfs}

\usepackage{tikz}
\usetikzlibrary{quotes, angles, intersections}
\newcommand{\R}{\mathbb{R}}
\newcommand{\D}{\mathscr{D}}
\newcommand{\Pp}{\mathscr{P}}
\newcommand{\Cc}{\mathscr{C}}
\newcommand{\E}{\mathscr{E}}

\newcommand{\bigO}{\mathscr{O}}
%\usetikzlibrary{matrix}
%\usepackage{pgfplots}
%\pgfplotsset{compat=1.4}
\usepackage{algorithm2e}
\usepackage{algorithmic}
\usepackage[utf8]{inputenc}
\usepackage{caption}
\newcommand{\source}[1]{\caption*{Source: {#1}} }

\begin{document}
	\section{Intro}
	Queremos encontrar uma ellipse com major-axis $a$ e minor-axis $b$, tal que ela contenha três pontos $(x_1, y_1), (x_2, y_2), (x_3, y_3)$.
	
	Vamos assumir que $(x_1, y_1) = (0,0)$ e $x_2 = 0$. Veja que podemos aplicar translação e rotação aos pontos originais para obter uma instância desta forma.
	
	A equação da ellipse é dada por
	
	\begin{equation}
	Ax^2 + Bxy + Cy^2 + Dx + Ey + F = 0
	\end{equation}
	
	Substituindo $(x_1, y_1)$ chegamos que $F=0$.
	
	Vamos assumir que todos os outros parametros são diferente de $0$. Veja que  as equações dadas por $(A, B, C, D, E)$ e $(\lambda A, \lambda B, \lambda C, \lambda D, \lambda E)$ representam a mesma elipse. Por isso vamos fixar $B=-1$.
	

	
	Substituindo $(0, y_2)$ obtemos:
	
	\begin{align*}
	Cy_2^2 + Ey_2 = 0\\
	E = -Cy_2
	\end{align*}
	
	Substituindo $(x_3, y_3)$ obtemos:
	
	\begin{align*}
	Ax_3^2 + x_3y_3 + C(y_3^2 - y_2y_3) + Dx_3 = 0\\
	D = -\frac{Ax_3^2 - x_3y_3 + C(y_3^2 - y_2y_3)}{x3}
	\end{align*}
	
	Agora nos resta fixar os parametros da forma da elipse $(a, b)$ e determinar $A$ e $C$.

	\section{Determinando $A, C$}
	
	Fixando $B=1$ impomos que o angulo de rotação da elipse seja diferente de $0$.
	
	Temos duas equações que determinam $a,b$. Seja $\Delta=4AC-B^2=4AC-1$.
	
	\begin{align}
	a^2 = \dfrac{2\dfrac{AE^2 -BDE +CD^2}{\Delta}}{A + C - \sqrt{1 + (A-C)^2}}\\
	b^2 = \frac{2\dfrac{AE^2 -BDE +CD^2}{\Delta}}{A + C + \sqrt{1 + (A-C)^2}}
	\end{align}
	
	Deixe que $q = \frac{a^2}{b^2}$, $q$ representa a ecentricidade ou a proporção entre $a$ e $b$. Fixando $q$, conseguimos fixar a forma, mas não a escala da elipse. Vamos fazer isso pois as contas ficam mais simples.
	
	\begin{align*}
	q = \frac{A + C + \sqrt{1 + (A-C)^2}}{A + C - \sqrt{1 + (A-C)^2}}\\
	\sqrt{1 + (A-C)^2}(q+1) = (A+C)(q-1)\\
	\end{align*}
	
	Seja $w = \frac{q-1}{q+1} = \frac{a^2-b^2}{a^2+b^2}$, veja que $w < 1$, isso será útil em algum momento. Portanto:
	
	\begin{align*}
	1 + (A-C)^2 = w^2(A+C)^2\\	
	\end{align*}
	
	Também vamos fixar o angulo de rotação $\theta \in [0, \frac{\pi}{4}]$. Seja $t = \tan{(2\theta)}$, temos:
	
	\begin{align*}
	t = \frac{B}{A-C}\\
	A = C - \frac{1}{t}
	\end{align*}
	
	Voltando a equação anterior, temos:
	
	\begin{align*}
	1 + (C - \frac{1}{t} - C)^2 = w^2(C-\frac{1}{t} + C)^2\\
	1 + \frac{1}{t^2} = w^2(2C-\frac{1}{t})^2
	\end{align*}
	
	Resolvendo a quadrática, achamos o valor de $C$:
	
	\begin{align}
	C = \frac{w \pm \sqrt{t^2+1}}{2tw}
	\end{align}
	
	Só vamos utilizar $C^+$ (o $C^-$ da um angulo de rotação maior que $\frac{\pi}{4}$, ainda nao sei o porquê).
	
	\subsection{Algumas propriedades}
	
	A condição $4AC - B^2 > 0$ tem que ser satisfeita para que a equação da conica represente uma elipse.
	
	Seja $\Gamma = A+C+\sqrt{1+(A-C)^2}$, substituindo $A$ em $\Gamma$, temos:
	
	\begin{align*}
	\Gamma = 2C - \frac{1}{t} + \sqrt{1 + \frac{1}{t^2}}
	\end{align*}
	
	Como estamos lidando primeiro com $t>0$, temos:
	
	\begin{align*}
	\Gamma = 2C - \frac{\sqrt{1+t^2} - 1}{t}
	\end{align*}
	
	Algumas propriedades que verifiquei utilizando Wolfram:
	
	\begin{itemize}
		\item $4AC - B^2>0$ faz com que $A>0$ e $C>0$.
		\item $C'<0$ para todo $t$.
		\item $A'<0$ se e só se $t < \dfrac{\sqrt{1-w^2}}{w}$
		\item $\Delta' < 0$ para todo $t$.
		\item $\Gamma>0$ e $\Gamma' < 0$ para todo $t$.
	\end{itemize}

%	Vamos agora olhar para $D$ e $E$:
	
	\begin{align*}
	D = -(A\alpha + \beta + C\gamma)\\
	E = -C\eta\\
	\alpha, \gamma, \eta>0
	\end{align*}
	
	Suas derivadas
	
	\begin{align*}
	D' = -(A'\alpha + C'\gamma)
	\end{align*}
	
	\subsection{Só Falta achar $t$}
	
	Infelizmente não é fácil achar $t$.
	
	\section{Outro Approach}
	
	Vamos supor que os pontos são $(0,0), (h, 0), (\alpha h, m \times \alpha h)$ e que $m$ é positivo.
	
	Temos que $D = -Ah$ e também fixamos $B=-1$.
	
	Vamos agora utilizar a equação que fixa a ecentricidade para isolar $C$.
	
	\begin{align*}
	1 + (A-C)^2 = w^2(A+C)^2\\
	C = \frac{-Aw^2 - A \pm \sqrt{4A^2w^2 + w^2 -1}}{w^2 - 1}
	\end{align*}
	
	Fixando o $A$ e encontrando $C$, talque $4AC > 1$ irá fixar a ecentricidade da elipse.
	
	\subsection{O terceiro ponto}
	
	Vamos olhar para a área detemrminada pelo triangulo dos pontos, chamamos a de $s$. Veja que o diametro da elipse que contem este triangulo é monótona em $s$ que por sua vez é monotona em $h$.
	
	Primeiro, econtramos a intersecção da reta $y=mx$ com a elipse.
	
	\begin{align*}
	Ax^2 -x^2m + Cx^2m^2 -Ahx + Emx = 0\\
	Ax -xm + Cxm^2 -Ah + Em = 0\\
	x(A-m + Cm^2) = Ah - Em\\
	x = \frac{Ah-Em}{A-m + Cm^2}
	\end{align*}
	
	Para determinar $E$, utilizamos o terceiro ponto.
	
	\begin{align*}
	A\alpha^2h^2 + B\alpha^2 h^2 m + Cm^2\alpha^2h^2 -Ah^2 \alpha + E m \alpha h = 0\\
	Em = -(	A(\alpha h - h) -\alpha hm + Cm^2h\alpha)
	\end{align*}
	
	Voltando para a equação anterior obtemos:
	
	\begin{align*}
	x = \frac{Ah+(	A(\alpha h - h) -\alpha hm + Cm^2h\alpha)}{A-m + Cm^2}\\
	y = mx\\
	y = m\frac{Ah+(	A(\alpha h - h) -\alpha hm + Cm^2h\alpha)}{A-m + Cm^2}
	\end{align*}
	
	Tomando a área do triangulo obtemos, veja que $s$ é conhecida dado $h$:
	
	\begin{align*}
	m\frac{Ah+(	A(\alpha h - h) -\alpha hm + Cm^2h\alpha)}{A-m + Cm^2} \times h = s\\
	A(mh^2\alpha) - h^2m^2\alpha + Cm^3h^2\alpha = s(A-m+Cm^2)\\
	A(mh^2\alpha -s) + C(m^3h^2\alpha -sm^2) = h^2m^2\alpha\\
	A\beta_1 + C\beta_2 + \beta_3 = 0
	\end{align*}
\end{document}