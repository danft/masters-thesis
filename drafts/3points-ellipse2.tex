\documentclass{article}

\usepackage{amsfonts}
\usepackage{amsmath}
\usepackage{mathrsfs}

\usepackage{tikz}
\usetikzlibrary{quotes, angles, intersections}
\newcommand{\R}{\mathbb{R}}
\newcommand{\D}{\mathscr{D}}
\newcommand{\Pp}{\mathscr{P}}
\newcommand{\Cc}{\mathscr{C}}
\newcommand{\E}{\mathscr{E}}

\newcommand{\bigO}{\mathscr{O}}
%\usetikzlibrary{matrix}
%\usepackage{pgfplots}
%\pgfplotsset{compat=1.4}
\usepackage{algorithm2e}
\usepackage{algorithmic}

\usepackage{caption}
\newcommand{\source}[1]{\caption*{Source: {#1}} }

\begin{document}

		
	\section{Inter}
	We have ellipses $E_0, E_1, E_2$, where $E_0$ is centered at $(0,0)$. Let the intersection points between $E_0$ and $E_i$ be $r_i$, we want $r1=r2$.
	
	Let $h_i,k_i$ be the center of $E_i$.
	
	we have the $r_i$
	
	\begin{align}
	y = \frac{h_1h_2^2 - h_2h_1^2 + q(h_1k_2^2 - h_2k_1^2)}{2q(h_1k_2-h_2k_1)}\\
	x_i = y_i \frac{-2qk_i}{2h_i} + \frac{h_i^2+qk_i^2}{2h_i}
	\end{align}
	
	Also, let's already use the rotation:
	
	\begin{equation}
	h_i=\cos{\theta}x_i - \sin{\theta}y_i, k_i=\sin{\theta}x_i + \cos{\theta}y_i
	\end{equation}
	
	Using the fact that $h_i^2 + k_i^2 = a^2 + b^2$ and simplifying (with wolfram) the denominator we have, we have:
	
	\begin{align*}
	y = \frac{(q-1)(h_2^2h_1 - h_1^2h_2) + q(a^2+b^2)(h_1-h_2)}{2q(x_1y_2-x_2y_1)}=\\
	y = \frac{(h_2-h_1)((q-1)h_2h_1 -q(a^2+b^2))}{2q(x_1y_2-x_2y_1)}
	\end{align*}
\end{document}