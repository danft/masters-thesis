Similarly to the method developed in \cite{church:1984} for the Euclidean PMCLP, we will describe a Candidate List Set (CLS) of possible locations for each ellipse and then propose an algorithm that constructs solutions combining the possible locations in each ellipse's CLS.
Also, based on the approach of \cite{church:1984} for the Euclidean PMCLP, and in \cite{chazelle:1986} for the problem of covering points by one unit disk, we will construct the CLS for each ellipse by using a property of the one-facility MCE.
Although, it is intuitive that it is possible to adapt the ideas used for solving the problems involving Euclidean disks for axis-parallel ellipses, using the results of \cite{bi}, which studies the problem of determining the intersection of $n$ strictly convex disks with fixed radius, we will prove that for MCE, we have all the necessary geometric properties to develop an algorithm following those approaches.

We start by introducing a lemma stating that any solution of the one-facility MCE can be related to the intersection of coverage regions of ellipses centered at the points covered in that solution. 

\begin{lem}\label{lema:mce-mwc}
	Let $(\Pp, \Ww, \{(a, b)\})$ be an instance of MCE, and $q\in \R^2$. If $\Pp \cap E(q) = A$, then $q \in \cap_{p\in A} E(p)$.
\end{lem}
\begin{proof}
	Let $u, v \in \R^2$, then we have that $u \in E(v)$ if, and only if $v \in E(u)$, which is proved simply by observing that:
	\begin{equation}\label{eq:peq}
	u \in E(v) \Leftrightarrow ||u-v||_{a, b, 0} \le 1 \Leftrightarrow v \in E(u).
	\end{equation}
	As $\Pp \cap E(q) = A$ can be written as: for all $p \in A$, $p \in E(q)$. Then, using \autoref{eq:peq}, we get that for all $p \in A$, $q \in E(p)$, which implies that $q \in \cap_{p\in A} E(p)$.
\end{proof}



%To define this equivalent problem, we need to first state an ellipse's property.
%Let $(\Pp, \Ww, \{(a, b)\})$ be an instance of MCE with one facility, and  $p, q \in \R^2$, we have that

%The equivalent problem is given by $n$ ellipses with shape parameters $(a, b)$ centered at $\Pp$. Let $q\in\R^2$ be a solution of MCE for one facility, by applying \autoref{eq:peq} to every point covered by $E(q)$, we obtain that
%\begin{equation}\label{eq:mce-mwc}
%\Pp \cap E(q) = \{p_i\in\Pp \colon q\in E(p_i)\},
%\end{equation}
%which implies that the problem of determining $q\in\R^2$ to maximize $w(\{p_i\in\Pp \colon q\in E(p_i)\})$, is equivalent to MCE for one facility. This changes the problem from determining a location for an ellipse given $n$ points to the problem of finding a point given $n$ ellipses with fixed locations.

%Let $A\subset \Pp$, from \autoref{eq:mce-mwc}, we have that if $\Pp \cap E(q) = A$ then $q \in \cap_{p\in A} E(p)$. 

From , if we could identify every 

Let $A \subset \Pp$, $|A|>1$, let us consider the intersection region $\cap_{p\in A} E(p) \neq \emptyset$. From \autoref{lema:mce-mwc}, we can also conclude that if $q \in \cap_{p\in A} E(p)$, then $A \subset \Pp \cap E(q)$.


In \cite{church:1984}, for the Euclidean PMCLP, it was proved that the border of this region contains at least one point that is an intersection between two fixed-radius circles with centers in $A$. This, as it turns out, is also true for ellipses, and a proof can be obtained from \cite{bi} where the problem of determining the intersection region of $n$ fixed-radius disks in a strictly convex normed plane is studied.
From that, we can conclude that there exists $u, v \in A$ and $q \in \partial E(u) \cap \partial E(v)$, such that $q \in \cap_{p\in A} E(p)$.
Before defining the CLS for each ellipse, we introduce a lemma stating two basic, and yet important properties about the intersection between two axis-parallel ellipses that have the same shape and distinct locations.

\begin{lem}\label{lema:e2p}
	Let $E$ be the coverage region of an axis-parallel ellipse with shape parameters $(a,b)$; and $v \in \R^2$, $v\neq0$. Then $|\partial E(0) \cap \partial E(v)| \le 2$, and $\partial E(0) \cap \partial E(v)$ can be determined analytically.
\end{lem}

\begin{proof}
	To determine the intersection points, consider the equality between the equations of $\partial E(0)$ and $\partial E(v)$:
	$x^2/a^2 + y^2/b^2 = (x-v_x)^2/a^2 + (y-v_y)^2/b^2.$
	This expression can be reduced to $y=\alpha x + \beta$, for some $\alpha, \beta$, which can then be plugged into $\partial E(0)$'s equation obtaining $x^2/a^2 + (\alpha x + \beta)^2/b^2 = 1$.
	We obtain the intersection points by solving this quadratic equation for $x$, and then, for each value of $x$,  determining $y$ from $y=\alpha x + \beta$.
\end{proof}
Next, we define the CLS of each ellipse, considering also the case where, in an optimal solution, an ellipse covers only one point.
\begin{definition}\label{def:cls_mce}
	Given an instance of MCE, for all $k \in \{1, \dots, m\}$, we define the CLS for the $k$-th ellipse as
	\begin{equation}
	S_k = \Pp \cup \left(\bigcup_{1 \le i < j \le n} \partial E_k(p_i) \cap \partial E_k(p_j) \right).
	\end{equation}
\end{definition}

By \autoref{lema:e2p}, the CLS for each ellipse can be computed in $\bigO(n^2)$, and $|S_k| \le n + 2\binom{n}{2}$. Next, we establish a lemma stating that the set of solutions obtained by combining the possible locations in each ellipse's CLS contains at least one optimal solution.

\begin{thm}\label{thm:mce}
	Given an instance of MCE, and $S_1, \dots, S_m$ as defined by \autoref{def:cls_mce}, then the set $\Omega = \{(q_1, \dots, q_m) \colon \textnormal{ for all }q_k \in S_k \}$ contains an optimal solution of MCE and $|\Omega| \le n^{2m}$. 
\end{thm}
\begin{proof}
	Let $Q^*$ be an optimal solution of MCE for the given instance. Then, we are going to prove that there exists $Q' \in \Omega$, which is also optimal.
	
	For each $k=1, \dots, m$, let $X_k = \{p_i \in \Pp\colon p_i \in E_k(q_k^*)\}$.
	
	%If $|X_k| = 0$, then any $q_k \in S_k$ makes $X_k \subset \Pp \cap E_k(q_k)$.
	
	If $|X_k| \le 1$, then there is at least one element in $S_k$ that makes $X_k \subset \Pp \cap E_k(q_k)$.
	
	If $|X_k| > 1$, then let $Y_k = \cap_{p \in X_k}E_k(p)$. By the results of \cite{bi}, we have that the boundary of $Y_k$ has vertices in the pairwise intersections of $\{\partial E_k(p) \colon p \in X_k\}$. Therefore, at least one vertex of $Y_k$ is in $S_k$, and any of those vertices produce a solution that covers at least the same points covered by $Q^*$.
	
	Lastly, we have that $|S_k| \le 2\binom{n}{2} + n = n(n+1)/2 \le n^2$. Hence, $|\Omega| \le n^{2m}$.
\end{proof}

With all this in hand, we define \autoref{algoritmo:mce}, which goes through every possible combination in the CLS of each ellipse. As evaluating each solution can be done in $\bigO(nm)$, we have that \autoref{algoritmo:mce} has $\bigO(mn^{2m+1})$ runtime complexity. 
In \autoref{section:numerical}, we give more details about the implementation of \autoref{algoritmo:mce} and analyze some numerical experiments for instances proposed in \cite{canbolat, andreta}, and for some new ones.

\begin{algorithm}
	\caption{Algorithm for MCE}\label{algoritmo:mce}
	
	\begin{algorithmic}[1]
		\Require{A set of points $\Pp=\{p_1,\dots,p_n\}$, a list of weights $\Ww=\{w_1, \dots, w_n\}$, and a list of shape parameters $\Rr=\{(a_1, b_1), \dots, (a_m, b_m)\}$.}
		
		\Ensure{An optimal solution for MCE.}
		
		\item[]
		\Procedure{$MCE$}{$\Pp, \Ww, \Rr$}
		\State \Return $MCE_{bt}(\Pp, \Ww, \Rr, 1)$
		\EndProcedure
		\State
		\Procedure{$MCE_{bt}$}{$Z, \Ww, \Rr, j$}
		\If{$j = |\Rr|+1$}
		\State \Return $0$
		\EndIf
		
		\State $(q_j^*, \dots, q_m^*) \gets (0, \dots, 0)$
		\State Let $S_j$ be the CLS for the $j$-th ellipse as defined by \autoref{def:cls_mce}.
		%\State $S_j \gets \textnormal{CLS-MCE}(Z, a_j, b_j)$
		\For{$q_j \in S_j$}
		\State $Cov \gets \Pp \cap E_j(q_j)$
		\State $(q_{j+1}, \dots, q_m) \gets MCE_{bt}(Z \setminus Cov, \Ww, \Rr, j+1)\}$
		
		\If{$w(\cup_{k=j}^m Z \cap E_k(q_k)) >  w(\cup_{k=j}^m Z \cap E_k(q_k^*))$}
		\State $(q_j^*, \dots, q_m^*) \gets(q_j, \dots, q_m)$
		\EndIf
		\EndFor
		
		\State \Return $(q_j^*, \dots, q_m^*)$
		\EndProcedure
	\end{algorithmic}
\end{algorithm}