We will develop a method which is similar to the one developed in \cite{drezner} for only one euclidean disk, and the exact algorithm developed for multiple Euclidean disks in \cite{cabello:2006}. 
We first describe a Candidate List set (CLS) for each facility, which is finite set of possible locations for each ellipse, which we use to converting MCE into a discrete optimization problem. We, then prove that using the possible solutions obtained from the combination of every ellipse's CLS an optimal solution can be obtained.

%\subsection{An equivalent problem}

Let $(\Pp, \Ww, \Rr)$ be an instance of MCE, then for each $j=1\dots m$, consider $n$ ellipses with shape parameters $(a_j, b_j)$ centered at each one of the points in $\Pp$. If we have $q_j\in\R^2$ in a subset of the coverage areas of those ellipses, then $q_j$ is a solution of MCE covering the centers of those ellipses. In other words, if $q_j \in \R^2$, and $X \subset \{1, \dots, n\}$, $X \neq \emptyset$, such that $q_j \in \cap_{i\in X}E_j(p_i)$, then $\Pp \cap E_j(q_j) = \{p_i \colon i \in X\}$.
From that observation, we can constrain each $q_j$ to be in $\cap_{i\in X} E_j(p_i)$, for some $X \subset \{1, \dots,n\}$, $X \neq \emptyset$.

In \cite{bi}, an algorithm is proposed for the problem of determining the intersection of disks of fixed radii from a strictly convex normed plane. 
We say that $(\R^2, ||\cdot||)$ is a strictly convex normed plane if the unit disk given by the norm $||\cdot||$ is strictly convex. Note that for any ellipse, there is a strictly convex normed plane whose unit circle is that ellipse.
For that reason, we state some results from \cite{bi} here, which we use on the development of an algorithm for MCE.


Let $(\R^2, ||\cdot||)$ be a strictly convex normed plane, and $\D=\{D_1, \dots, D_n\}$ be a set of $n$ unit disks in that space centered at different points, with the condition that  $\cap_{i=1}^n D_i \neq \emptyset$. In \cite{bi}, an algorithm was developed to construct this intersection in $\bigO(n\lg{n})$, some of its preliminary results are:
\begin{itemize}
	\item $\partial \cap_{i=1}^n D_i$ is formed by arcs of $\partial D_1, \dots, \partial D_n$.
	\item The vertices of $\partial \cap_{i=1}^n D_i$ is contained in the set of pairwise-intersection of the circles $\partial D_1, \dots, \partial D_n$.
	\item $|\partial D_i \cap \partial D_j| \le 2$.
\end{itemize}

Based on those, we introduce the next definition for the $k$-th ellipse's CLS, which we refer to as $S_k$.

\begin{definition}\label{def:cls_mce}
	Given an instance of MCE, for any $k \in \{1, \dots, m\}$, we define the CLS for the $k$-th ellipse as
	\begin{equation}
	S_k = \bigcup_{1 \le i < j \le n} \partial E_k(p_i) \cap \partial E_k(p_j) \bigcup \Pp.
	\end{equation}
\end{definition}

The set of solutions $S_k$, can be computed in $\bigO(n^2)$ as determining the intersections between two axis-parallel ellipses can be done analytically. 

\begin{lem}
	Given an instance of MCE, and $S_1, \dots, S_m$ as defined by \autoref{def:cls_mce}, then the set $\Omega = \{(q_1, \dots, q_m) \colon \textnormal{ for all }q_k \in S_k \}$ contains an optimal solution of MCE and $|\Omega| \le n^{2m}$. 
\end{lem}
\begin{pf}
	Let $Q^*$ be an optimal solution of MCE for the given instance. Then, we are going to prove that there exists $Q' \in \Omega$, which is also optimal.
	
	For each $k=1, \dots, m$, let $X_k = \{p_i \in \Pp\colon p_i \in E_k(q_k^*)\}$.
	
	If $|X_k| = 0$, then any $q_k \in S_k$ makes $X_k \subset \Pp \cap E_k(q_k)$.
	
	if $|X_k| = 1$, then there is at least one element in $S_k$ that makes $X_k \subset \Pp \cap E_k(q_k)$.
	
	if $|X_k| > 1$, then let $Y_k = \cap_{p \in X_k}E_k(p)$, by the results of \cite{bi}, we have that the boundary of $Y_k$ has vertices in the pairwise intersections of $\{\partial E_k(p) \colon p \in X_k\}$. Therefore, at least one vertex of $Y_k$ is in $S_k$, and any of those vertices produce a solution that covers at least the same points covered by $Q^*$.
	
	Lastly, we have that $|S_k| \le 2\binom{n}{2} + n = n(n+1)/2 \le n^2$ hence $|\Omega| \le n^{2m}$.
\end{pf}

With all this in hand, we can go ahead and define an algorithm for MCE.

\begin{algorithm}
	\caption{Algorithm for MCE}\label{algoritmo:mce}
	
	\begin{algorithmic}[1]
		\Require{A set of points $\Pp=\{p_1,\dots,p_n\}$, a list of weights $\Ww=\{w_1, \dots, w_n\}$, and a list of shape parameters $\Rr=\{(a_1, b_1), \dots, (a_m, b_m)\}$.}
		
		\Ensure{An optimal solution for MCE.}
		
		\item[]
		\Procedure{$MCE$}{$\Pp, \Ww, \Rr$}
		\State \Return $MCE_{bt}(\Pp, \Ww, \Rr, 1)$
		\EndProcedure
		\State
		\Procedure{$MCE_{bt}$}{$Z, \Ww, \Rr, j$}
		\If{$j = m+1$}
		\State \Return $0$
		\EndIf
		
		\State $(q_j^*, \dots, q_m^*) \gets (0, \dots, 0)$
		\State Let $S_j$ be the CLS for the $j$-th ellipse as defined by \autoref{def:cls_mce}.
		%\State $S_j \gets \textnormal{CLS-MCE}(Z, a_j, b_j)$
		\For{$q_j \in S_j$}
		\State $Cov \gets \Pp \cap E_j(q_j)$
		\State $(q_{j+1}, \dots, q_m) \gets MCE_{bt}(Z \setminus Cov, \Ww, \Rr, j+1)\}$
		
		\If{$w(\cup_{k=j}^m Z \cap E_k(q_k)) >  w(\cup_{k=j}^m Z \cap E_k(q_k^*))$}
		\State $(q_j^*, \dots, q_m^*) \gets(q_j, \dots, q_m)$
		\EndIf
		\EndFor
		
		\State \Return $(q_j^*, \dots, q_m^*)$
		\EndProcedure
	\end{algorithmic}
\end{algorithm}

\autoref{algoritmo:mce} can be proved to have a runtime complexity of $\bigO(mn^{2m+1})$.