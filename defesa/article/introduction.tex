The Planar Maximum Covering Location Problem (PMCLP) was first introduced in \cite{church:1984}, and can be seen as a category of problems where the coverage of a demand set, a collection of subsets of $\R^2$, is to be maximized by determining the location of facilities in $\R^2$, with coverage being determined by a distance function.
In \cite{church:1984}, methods for Euclidean and Rectilinear distances versions of the problem were proposed.
In \cite{drezner, chazelle:1986}, exact algorithms for the Euclidean PMCLP with only one facility are proposed; and in \cite{cabello:2006} an approximation algorithm is proposed for the version with multiple unit disks as facilities.
A property of the Euclidean PMCLP, which is utilized in the algorithms developed in \cite{drezner, chazelle:1986, cabello:2006}, and in the method proposed in \cite{church:1984}, is that there is an optimal solution which every facility is located in the demand points, or in the intersection of two circles centered at two demand points; we will prove a similar property for ellipses in our work.

It is fair to say that PMCLP with elliptical coverage has not been vastly studied as only two articles have been found on it. In \cite{canbolat}, a mixed-integer non-linear programming method was proposed as a first approach to the problem, and, because it took too long and did not find an optimal solution for some of the proposed instances, a heuristic method was developed using a technique called Simulated Annealing. 
%For some instances, the method took too long and did not find an optimal solution. Because of that, a heuristic method was developed using a technique called Simulated Annealing, which was able to obtain solutions for the instances proposed in that study.

Initially, in the version of the problem defined in \cite{canbolat}, the ellipses could not be rotated. In \cite{andreta} though, the problem where the ellipses can be freely rotated was introduced.

which introduced the version where the ellipses can be freely rotated. 

to which an exact and a heuristic method was proposed, and developed a new method for the axis-parallel version of the problem, which was able to obtain optimal solutions for instances that the method proposed by \cite{canbolat} could not.
The exact method for the version with rotation could not obtain optimal solutions within a predefined time limit for several instances, the heuristic method though returned solutions for every instance, and impressively enough, obtained optimal solutions for every verifiable instance.

We study two versions of PMCLP with elliptical coverage facilities in this work. For both of them, each ellipse is defined to have a fixed shape and an undefined location, which is part of the solution.
In the first version, introduced in \cite{canbolat}, all the ellipses are restricted to be axis-parallel, while in the second version, introduced in \cite{andreta}, this constraint is dropped, and all the ellipses can be freely rotated.
The first version will be referred to as Planar Maximum Covering Location by Ellipses Problem (MCE) and the second one as  Planar Maximum Covering Location by Ellipses with Rotation Problem (MCER).

	
