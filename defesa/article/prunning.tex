Without any improvement, backtracking through every possible combination of every ellipse's CLS can take a very long time, possibly going through a lot of non-optimal solutions. 
For this reason, we introduce a sufficient condition for the MCER's case (the MCE's case is analogous), based on MCER for one ellipse, which can be used to skip solutions that for sure are non-optimal.

\begin{definition}
	Given an instance $(\Pp, \Ww, \Rr)$ of MCER. We define $OPT_j$ as the value of the best solution with the first $j$ ellipses fixed at locations $(q_1, \theta_1); \dots; (q_j, \theta_j)$, and $Z_j=\Pp \setminus \cup_{k=1}^j E_k(q_k, \theta_k)$.
\end{definition}

Then, we can obtain an upper-bound for $OPT_j$ by using, for each $k\in\{j+1, \dots, m\}$, the solution $(q_k', \theta_k')$ of MCER for instance $(Z_k, \{w_i\colon p_i \in Z_k\}, \{(a_k, b_k)\})$. As these solutions only consider the best cover individually for each ellipse, we have the following inequality
\begin{equation*}
OPT_j \le w\left(\bigcup_{k=1}^{j} \Pp \cap E_k(q_k, \theta_k)\right) + w\left(\bigcup_{k=j+1}^{m} \Pp \cap E_k(q_k', \theta_k')\right).
\end{equation*}
This upper-bound for $OPT_j$ can then be used in the backtracking process to skip solutions that are not better than any optimal solution. Let $OPT_{lo}$ be a lower bound for the optimal solution, we have that if
\begin{equation}\label{eq:upper-bound}
w\left(\bigcup_{k=1}^{j} \Pp \cap E_k(q_k, \theta_k)\right) +w\left(\bigcup_{k=j+1}^{m} \Pp \cap E_k(q_k', \theta_k')\right) \le OPT_{lo},
\end{equation}
then $OPT_j \le OPT_{lo}$, which implies that $OPT_j$ is less than or equal the value of any optimal solution. This defines a sufficient condition for us to dismiss every solution which have the location of the first $j$ ellipses fixed at $(q_1, \theta_1); \dots; (q_j, \theta_j)$. In practice, we can use the value of the best solution found so far as the lower-bound for the optimal solution.

It is worth pointing out that these improvement suggestions do not have an effect in a possible worst case scenario. We are adopting them in our implementation because they showed good results in practice.
For example, without taking the suggestion given by \autoref{eq:upper-bound}, 
MCER-$k$'s algorithm takes nine seconds to obtain an optimal solution for instance AB060, going through \num{336494451} solutions, while the implementation using \autoref{eq:upper-bound} to prune the backtracking tree for the same instance takes less than one second, and evaluates only \num{1809} solutions in total.
