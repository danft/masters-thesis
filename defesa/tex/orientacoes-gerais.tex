
\section{Codificação dos arquivos: UTF8}

A codificação de todos os arquivos do pacote \abnTeX, incluindo a classe \textit{icmc}, é \texttt{UTF8}. É necessário que
você utilize a mesma codificação nos documentos que escrever, inclusive nos
arquivos de base bibliográficas |.bib|.



\section{Inclusão de outros arquivos}\label{sec-include}

É uma boa prática dividir o seu documento em diversos arquivos, e não
apenas escrever tudo em um único. Para tanto, esse recurso foi utilizado neste
documento, além de estarem organizados em um diretório separado do arquivo principal. Para incluir diferentes arquivos em um arquivo principal,
de modo que cada arquivo incluído fique em uma página diferente, utilize o
comando:

\begin{verbatim}
   \include{tex/documento-a-ser-incluido}    % sem a extensão .tex
\end{verbatim}

Para incluir documentos sem quebra de páginas, utilize:

\begin{verbatim}
   \input{tex/documento-a-ser-incluido}      % sem a extensão .tex
\end{verbatim}



\section{Remissões internas}

Ao nomear a \autoref{tab:lista_produtos} e a \autoref{fig:logomarca_usp}, apresentamos um exemplo de remissão interna, que também pode ser feita quando indicamos o \autoref{chapter:corpos-flutuantes}, que tem o nome \emph{\nameref{chapter:corpos-flutuantes}}. O número do capítulo indicado é \ref{chapter:corpos-flutuantes}, que se inicia à \autopageref{chapter:corpos-flutuantes}\footnote{O número da página de uma remissão pode ser obtida também assim:
\pageref{chapter:corpos-flutuantes}.}.

O código usado para produzir o texto desta seção é:

\begin{verbatim}
Ao nomear a \autoref{tab:lista_produtos} e a \autoref{fig:logomarca_usp}, 
apresentamos um exemplo de remissão interna, que também pode ser feita 
quando indicamos o \autoref{chapter:corpos-flutuantes}, que tem o nome 
\emph{\nameref{chapter:corpos-flutuantes}}. O número do capítulo indicado 
é \ref{chapter:corpos-flutuantes}, que se inicia à 
\autopageref{chapter:corpos-flutuantes}
\footnote{O número da página de uma remissão pode ser obtida também assim: 
\pageref{chapter:corpos-flutuantes}.}.
\end{verbatim}

\section{Diferentes idiomas e hifenizações}
\label{sec-hifenizacao}

Para usar hifenizações de diferentes idiomas, inclua nas opções do documento o nome dos idiomas que o seu texto contém. Por exemplo (para melhor visualização, as opções foram quebras em diferentes linhas):

\begin{verbatim}
    \documentclass[
        qualificacao,
        mestrado
        pos-defesa,
        english,
        french,
        spanish,
        brazil
    ]{icmc}
\end{verbatim}

Os idiomas português-brasileiro (\texttt{brazil}) e inglês (\texttt{english}) são incluídos automaticamente pela classe \textsf{icmc}. Caso deseje utilizar outros idiomas no corpo do documento, como em citações em francês, você deve usar o preâmbulo como abaixo:

\begin{verbatim}
    \documentclass[
        doutorado,
        pre-defesa,
        french
    ]{icmc}
\end{verbatim}

A lista completa de idiomas suportados, bem como outras opções de hifenização, estão disponíveis em \citeonline[p.~5-6]{babel}.


\section{Comandos auxiliares úteis}

A classe \textit{icmc} contém alguns comandos auxiliares definidos com o objetivo de tornar o processo de escrita mais eficiente. Os principais comandos são apresentados a seguir:

\begin{description}
    
    \item[\comando{aspas\{CONTENT\}}] Comando utilizado para inserir um texto entre aspas.
    \item[\comando{autoref\{LABEL\}}] Comando utilizado para fazer referência a um elemento do texto. O parâmetro \texttt{LABEL} utilizado refere-se ao código definido por meio do comando \comando{label\{\}}.
    \item[\comando{fadaptada[CONTENT]\{REF\}}] Comando utilizado nos ambientes de \texttt{Figura}, \texttt{Tabela}, entre outros, para definir a origem da fonte do dado apresentado que foi adaptado de alguma referência. Os parâmetros utilizados são: \texttt{REF} que é o índice da referência no arquivo bibtex, e; \texttt{CONTENT} que é a localização exata do dado na referência (Ex.: p~30). O parâmetro \texttt{CONTENT} é opcional.
    \item[\comando{fautor}] Comando utilizado nos ambientes de \texttt{Figura}, \texttt{Tabela}, \texttt{Quadro}, entre outros, que define o próprio autor como provedor da informação.
    \item[\comando{fdadospesquisa}] Comando utilizado nos ambientes de \texttt{Figura}, \texttt{Tabela}, \texttt{Quadro}, entre outros, que define que os dados originaram da própria pesquisa.
    \item[\comando{fdireta[CONTENT]\{REF\}}] Comando utilizado nos ambientes de \texttt{Figura}, \texttt{Tabela}, \texttt{Quadro}, entre outros, para definir a origem da fonte do dado apresentado que foi adaptado de alguma referência. Os parâmetros utilizados são: \texttt{REF} que é o índice da referência no arquivo bibtex, e; \texttt{CONTENT} que é a localização exata do dado na referência (Ex.: p~30). O parâmetro \texttt{CONTENT} é opcional.
    \item[\comando{newword\{WORD\}\{DESC\}}] Comando utilizado para inserir palavras no glossário de modo mais prático. Os parâmetros utilizados são: \texttt{WORD} que é a palavra que será descrita, e; \texttt{DESC} que é o significado da palavra.
    \item[\comando{rev\{CONTENT\}}] Comando utilizado para inserir notas de revisão dentro do texto, as quais aparecerão destacadas em vermelho. O parâmetro utilizado é \texttt{CONTENT} que contém o texto sobre a revisão.
    \item[\comando{sigla\{ABBR\}\{DESC\}}] Comando utilizado para inserir siglas e abreviaturas na Lista de siglas de modo mais prático. Os parâmetros utilizados são: \texttt{ABBR} que é a abreviatura ou sigla, e \texttt{DESC} sua descrição. Ao utilizar esse comando, a sigla \textbf{também} é inserida no texto do documento.
    \item[\comando{sigla*\{ABBR\}\{DESC\}}] Comando utilizado para inserir siglas e abreviaturas na Lista de Siglas de modo mais prático. Os parâmetros utilizados são: \texttt{ABBR} que é a abreviatura ou sigla, e \texttt{DESC} sua descrição. Ao utilizar esse comando, a sigla é inserida \textbf{apenas} na Lista de Siglas.
    \item[\comando{simbolo\{SYM\}\{DESC\}}] Comando utilizado para inserir símbolos na Lista de Símbolos de modo mais prático. Os parâmetros utilizados são: \texttt{SYM} que é o símbolo, e \texttt{DESC} sua descrição. 
    
\end{description}


\section{Consulte o manual da classe \textsf{abntex2}}

Consulte o manual da classe \textsf{abntex2} \cite{abntex2classe} para uma
referência completa das macros e ambientes disponíveis. 

Além disso, o manual possui informações adicionais sobre as normas ABNT
observadas pelo \abnTeX\ e considerações sobre eventuais requisitos específicos, como o caso da \citeonline[seção 5.2.2]{NBR14724:2011}, que
especifica o espaçamento entre os capítulos e o início do texto.



\section{Precisa de ajuda?}

Consulte a FAQ com perguntas frequentes e comuns no portal do \abnTeX:
\url{https://code.google.com/p/abntex2/wiki/FAQ}.

Inscreva-se no grupo de usuários \LaTeX:
\url{http://groups.google.com/group/latex-br}, tire suas dúvidas e ajude
outros usuários.

Participe também do grupo de desenvolvedores do \abnTeX:
\url{http://groups.google.com/group/abntex2} e faça sua contribuição à
ferramenta.


\section{Você pode ajudar?}

Sua contribuição é muito importante! Você pode ajudar na divulgação,
desenvolvimento, aprimoramento e de várias outras formas. Veja como contribuir com a classe \textit{icmc} em
\url{https://github.com/lordantonelli/thesis-model-icmc} e faça sua contribuição.
