A configuração de diversas opções e principalmente dos elementos pré-textuais é realizada com comandos específicos inseridos antes do comando \comando{begin\{document\}}. As informações do documento são configuradas através dos comandos:

\begin{description}

 \item[\comando{tituloPT\{T\}}] Título do trabalho em português (substitua T pelo título do trabalho em português);
 
 \item[\comando{tituloEN\{T\}}] Título do trabalho em inglês (substitua T pelo título do trabalho em inglês);

 \item[\comando{autor[REF]\{N\}}] Nome do autor do trabalho (onde REF é como o nome do autor é referenciado e N é o nome do autor);
 
  \item[\comando{genero\{GEN\}}] Gênero do autor. Substitua GEN pela sigla do gênero correspondente (M = Masculino ou F = Feminino);

 \item[\comando{orientador\{T\}\{O\}}] Nome do professor orientador do trabalho. Caso seja uma orientadora pode ser usado o comando \comando{orientador[Orientadora]\{T\}\{O\}} (sendo que T é a titulação do professor e O é o nome do orientador);

 \item[\comando{coorientador\{T\}\{C\}}] Nome do professor coorientador do trabalho. Caso seja uma coorientadora pode ser usado um comando análogo a definição de orientadora  empregando o comando \comando{coorientador[Coorientadora]\{T\}\{C\}}(sendo que T é a titulação do professor e C é o nome do orientador);

 
 \item[\comando{curso\{SPPG\}}] Dados do programa de Pós-Graduação, onde SPPG é a sigla do programa de pós-graduação. Exemplo: \comando{curso\{CCMC\}}. Os seguintes programas de Pós-Graduação estão disponíveis e configurados neste \textit{template}:
    \begin{itemize}
        \item \textbf{CCMC} -- Programa de Pós-Graduação em Ciências de Computação e Matemática Computacional
        \item \textbf{MAT} -- Programa de Pós-Graduação em Matemática
        \item \textbf{PIPGES} -- Programa Interinstitucional de Pós-Graduação em Estatística
        \item \textbf{PROFMAT} -- Programa de Pós-Graduação em Mestrado Profissional em Matemática em Rede Nacional
        \item \textbf{MECAI} -- Programa de Pós-Graduação em Mestrado Profissional em Matemática, Estatística e Computação Aplicadas à Indústria
    \end{itemize}
 
 \item[\comando{data\{dia\}\{mês\}\{ano\}}] Configuração da data do depósito do documento;
 
 \item[\comando{idioma\{LANG\}}] Definição do idioma principal que o documento será escrito. As opções disponíveis são: \textbf{PT} (para escrita em português) e \textbf{EN} (para escrita em inglês), que devem obrigatoriamente serem informadas em letras maiúsculas.

 \item[\comando{textoresumo\{TR\}\{PC\}}] Texto do resumo (TR) e palavras-chave (PC) do documento sendo separadas por vírgula. Se o idioma do resumo for diferente do declarado no documento, pode ser usado o comando \comando{textoresumo[L]\{TR\}\{PC\}} (sendo que L é a linguagem do resumo);
 
 \item[\comando{incluifichacatalografica\{ARQ\}}] Inclusão da ficha catalográfica do documento gerada diretamente no site da biblioteca \url{http://www.icmc.usp.br/Portal/Sistemas/Biblioteca/ficha.php}, em que ARQ é o nome do arquivo PDF, incluindo o caminho do diretório se necessário.

 
\end{description}

As opções seguintes correspondem também as configurações dos elementos pré-textuais, porém seu uso é opcional: 

\begin{description}

 \item[\comando{textodedicatoria\{TD\}}] Texto referente a dedicatória do trabalho (TD). Caso o texto esteja em um arquivo separado (recomendado para que o projeto fique modularizado e os documentos mais limpo), deve utilizar o comando \comando{textodedicatoria*\{ARQ\}}, em que ARQ é o nome do arquivo, incluindo o caminho do diretório se necessário;

 \item[\comando{textoagradecimentos\{TA\}}] Texto referente aos agradecimentos do trabalho (TA). Caso o texto esteja em um arquivo separado (recomendado para que o projeto fique modularizado e os documentos mais limpo), deve utilizar o comando \comando{textoagradecimentos*\{ARQ\}}, em que ARQ é o nome do arquivo, incluindo o caminho do diretório se necessário;

 \item[\comando{incluilistadefiguras}] Comando para inclusão da lista de figuras. Deve-se utilizar este comando somente quando o ambiente \textbf{figure} for utilizado no documento;
 
 \item[\comando{incluilistadetabelas}] Comando para inclusão da lista de tabelas. Deve-se utilizar este comando somente quando o ambiente \textbf{table} for utilizado no documento;
  
 \item[\comando{incluilistadequadros}] Comando para inclusão da lista de quadros. Deve-se utilizar este comando somente quando o ambiente \textbf{quadro} for utilizado no documento;
   
 \item[\comando{incluilistadealgoritmos}] Comando para inclusão da lista de algoritmos. Deve-se utilizar este comando somente quando o ambiente \textbf{algoritmo} for utilizado no documento;
    
 \item[\comando{incluilistadecodigos}] Comando para inclusão da lista de figuras. Deve-se utilizar este comando somente quando o ambiente \textbf{codigo} for utilizado no documento;
 
 \item[\comando{incluilistadesiglas}] Comando para inclusão da lista de siglas e abreviaturas. Deve-se utilizar este comando somente quando existirem siglas e abreviaturas no documento, com a utilização do comando \comando{sigla\{S\}\{DS\}} ou \comando{sigla*\{S\}\{DS\}};

 \item[\comando{incluilistadesimbolos}] Comando para inclusão da lista de símbolos. Deve-se utilizar este comando somente quando existirem símbolos no documento, com a utilização do comando \comando{simbolo\{S\}\{DS\}}.
 
\end{description}