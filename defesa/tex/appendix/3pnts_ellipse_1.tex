Let $p_1, p_2, p_3 \in \R^2$ and $E$ be an ellipse with shape parameters $(a, b) \in \R^2_{>0}$, with $a > b$. The problem is to find every pair $(q, \theta)$, with $q \in \R^2$ and $\theta \in [0, \pi]$, such that $\{p_1, p_2, p_3\} \subset \tilde{E}(q, \theta)$. In other words, one wants to find every center and angle of rotation of $E$, such that $p_1$, $p_2$ and $p_3$ are on the border of $E$.
The goal of this section is to show that the number of solutions of such problem is limited, finding every one of them analytically is beyond the scope of this work.

\section{Transforming the input}

To make it simpler, let us translate the system to put $p_1$ at $(0,0)$. Then, we assume that the ellipse is actually axis-parallel and the points are the ones rotating. When an angle is found such that the three points lie on the border of the ellipse, a linear transformation can be applied to compress the x-axis by $\frac{b}{a}$, transforming the ellipse into a circle of radius $b$. This is described by \autoref{eq:trpnts}.

\begin{equation}\label{eq:trpnts}
\varphi(p)=\left[\begin{array}{cc}
\frac{b}{a}&0\\
0&1
\end{array}\right]
\left[\begin{array}{cc}
\cos{\theta}&\sin{\theta}\\
-\sin{\theta}&\cos{\theta}
\end{array}\right]\left[\begin{array}{c}
p_x\\
p_y
\end{array}\right]
\end{equation}

The only unknown variable in \autoref{eq:trpnts} is $\theta$. Instead of dealing with trigonometric equations, we will use the following relations:

\begin{equation}\label{eq:trigId}
	\begin{array}{c}
\cos{\theta} = \dfrac{t^2-1}{t^2+1}\\
\sin{\theta} = \dfrac{2t}{t^2+1}.
\end{array}
\end{equation}

After replacing the trigonometric functions with \autoref{eq:trigId}, we get:

\begin{equation}
\varphi(p)=\left[\begin{array}{cc}
\dfrac{b}{a}\dfrac{t^2-1}{t^2+1}& \dfrac{b}{a}\dfrac{2t}{t^2+1}\\
\dfrac{t^2-1}{t^2+1}& -\dfrac{2t}{t^2+1}\\
\end{array}\right]p.
\end{equation}

Thus $\varphi(p_1), \varphi(p_2), \varphi(p_2)$ can be written as quotients of univariate polynomials of degree $2$.

\section{Three points on a circle}
%find ref%
There is a known way to determine the circle which contains three given points. 
Let $(x_1, y_1)$, $(x_2, y_2)$, $(x_3, y_3)$ be three non-colinear points in $\R^2$, the equation of the circle passing through these points is given by the determinant on \autoref{3pnt_circle}

\begin{equation}\label{3pnt_circle}
\left|
\begin{array}{cccc}
x^2+y^2&x&y&1\\
x_1^2+y_1^2&x_1&y_1&1\\
x_2^2+y_2^2&x_2&y_2&1\\
x_3^2+y_3^2&x_3&y_3&1
\end{array}
\right| = 0.
\end{equation}

As for this problem $(x_1, y_1)=(0,0)$, we can rewrite \autoref{3pnt_circle} as:

\begin{equation}
\left|
\begin{array}{ccc}
x^2+y^2&x&y\\
x_2^2+y_2^2&x_2&y_2\\
x_3^2+y_3^2&x_3&y_3
\end{array}
\right| = 0.
\end{equation}

Expanding the determinant we get:

\begin{equation*}
(x^2+y^2)\left|
\begin{array}{cc}
x_2&y_2\\
x_3&y_3
\end{array}\right|
+
x\left|
\begin{array}{cc}
x_2^2+y_2^2&y_2\\
x_3^2+y_3^2&y_3
\end{array}\right|
+
y\left|
\begin{array}{cc}
x_2&x_2^2+y_2^2\\
x_3&x_3^2+y_3^2
\end{array}\right|=0.
\end{equation*}

After completing the square we obtain \autoref{eq:le3} which is an expression for the radius of the circle, which has to be equal $b$.

\begin{equation}\label{eq:le3}
\dfrac{\left|
	\begin{array}{cc}
	x_2^2+y_2^2&y_2\\
	x_3^2+y_3^2&y_3
	\end{array}\right|^2
+\left|
\begin{array}{cc}
x_2&x_2^2+y_2^2\\
x_3&x_3^2+y_3^2
\end{array}\right|^2
}{4\left|
\begin{array}{cc}
x_2&y_2\\
x_3&y_3
\end{array}\right|^2} = b^2.
\end{equation}

We want to show that \autoref{eq:le3}, after replacing $(x_2, y_2)$ and $(x_3, y_3)$ with $\varphi(p_2)$ and $\varphi(p_3)$, is a polynomial on $t$ of degree $12$. As the denominator of $\varphi(p)_x$ and $\varphi(p)_y$ are the same, we can ignore it as it cancels out when it is equal to $0$. Let us analyze one of the determinants from the numerator of \autoref{eq:le3} which is the term with the highest degree.

\begin{equation}\label{eq:num1}
\left|
	\begin{array}{cc}
	||\varphi(p_2)||^2&\varphi(p_2)_y\\
	||\varphi(p_3)||^2&\varphi(p_3)_y
	\end{array}\right|^2=(||\varphi(p_2)||^2\varphi(p_3)_y-||\varphi(p_3)||^2\varphi(p_2)_y)^2.
\end{equation}

As $||\varphi(p_2)||^2$ has degree $4$ and $\varphi(p_3)_y$ has degree $2$ (ignoring the denominator). After multiplying and squaring we arrive at the conclusion that the polynomial has degree $12$.

\section{Final Remarks}

Expressing the actual polynomial would take too much space, and actually finding every one of its roots is a very expensive task computationally, so just determining its degree was enough for what we wanted to show.

Also, ellipses are symmetrical with respect to their major-axis which means that an ellipse is identical to itself rotate by $\pi$. Because of that, any solution found has a symmetrical one, which is obtained by adding $\pi$ to it. This halves the upper-bound for the number of solutions to $6$.