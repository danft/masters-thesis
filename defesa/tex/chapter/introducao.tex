The Minimum Cover Problem --also known as just the Set Cover Problem--, and the Maximal Covering Problem are the two main types of optimal covering problems found in the literature \cite{karatas}. 

One of the 21 Karp's NP-Complete problems\footnote{The decision version, which asks if there is a cover of size $k$, is NP-Complete.} \cite{karp}, the Minimum Cover Problem is very well explored and considered to be a classic. 
Given a demand set along with a collection of subsets of the demand set. The problem is then to determine the minimum number of elements from the collection of subsets that are needed to cover the whole demand set. One of its most famous examples is the Minimum Vertex Cover defined over graphs, where the vertex set has to be covered by a subset of edges.

The second type of covering problems arose from the fact that covering almost all the demand set can be a lot cheaper than having to cover it all \cite{garcia}. This second type is known as \sigla{\MCLP}{Maximum Covering Location Problem}\footnote{Originally, the problem was defined using the word \textit{Maximal} on its title. In some sources, however, the word \textit{Maximum} is adopted.} and was introduced in \citeonline{church:1974}.
In this first study, the author defined the problem for graphs, such that the demand set is a subset of the graph's vertices. Given a facility set where each facility has a coverage radius, the problem asks to choose vertices to place the facilities to maximize the coverage of the demand set.

Just like the Minimum Cover problem, MCLP is NP-Hard \cite{hatta:2013} and both deterministic, using integer programming in \citeonline{church:1974}; and heuristic methods in \citeonline{revelle:2008} have been proposed to solve it. A very complete survey of developments in this area can be found in \citeonline{mclp_review}.

In \citeonline{church:1984} a new kind of MCLP named \sigla{\PMCLP}{Planar Maximum Covering Location Problem} was introduced. Unlike its predecessor, this version of the problem was not defined on a network. Instead, on PMCLP the demand set and the facilities are located in $\R^2$ and a facility's coverage area is determined by a distance function. 
Initially, the Euclidean distance was considered and the idea behind the method proposed in \citeonline{church:1984} was to convert PMCLP into an instance of MCLP and then utilize any of the previous developed exact methods to obtain a solution for PMCLP. This reduction was done by identifying a \sigla{CLS}{Candidate Locations Set} which represented the possible locations that needed to be evaluated for every facility, such that the optimal solution could be found. From the CLS, a network was built on which MCLP could be applied. Generating the CLS specifically for the case of Euclidean distance will be described here on \autoref{chapter:pmclp}.

Furthermore, some variations of PMCLP can also be found in the literature: in \citeonline{younies} PMCLP was studied under the block norm distance, in \citeonline{mumtaz} a mean-shift algorithm for large scale\footnote{Numerical experiments were done for up to $3000$ points.} PMCLP was proposed, and in \citeonline{bansal} a version with partial coverage and rectangular demand and facility zones was introduced.

PMCLP under Euclidean distance is also found in the literature as the \sigla{MCD}{Maximum Covering by Disks} problem.
Early works only tackled the one-disk version of it. In \citeonline{chazelle:1986} a $\bigO(n^2)$ algorithm, which still stands as the best in terms of run-time complexity, was proposed beating the prior $\bigO(n^2\log{n})$ algorithm created by \citeonline{drezner}.
The $m$ disks version of MCD was studied in \citeonline{cabello:2006} which had as its most important result a $(1-\epsilon)$-approximation algorithm which runs in $\bigO(n\log{n})$. To achieve its main goal, however, they developed a deterministic $\bigO(n^{2m-1}\log{n})$ algorithm which gets employed into their approximation scheme.
Additionally, in \citeonline{aronov:2008} one-disk maximum covering is proven to be 3SUM-HARD. This means that maximizing the number of points covered by a disk is as hard as finding three real numbers that sum to zero among $n$ given real numbers.

Two versions of PMCLP are studied in this work, both of them related to ellipses. The first one was introduced in \citeonline{canbolat} and will be referred to here as \sigla{PMCE}{Planar Maximum Covering by Ellipses}. This version does not allow the ellipses to rotate, so basically it only differs from its disks counterpart in the shape of the facility's coverage area. The second version was introduced in \citeonline{andreta} and will be referred to here as \sigla{PMCER}{Planar Maximum Covering by Ellipses with Rotation}. This version of the problem had no restriction on the rotation of ellipses--note that this is a slightly bigger modification of PMCLP compared to the first one, as not only the location for the facilities must be determined, but also their angle of rotation.
The main motivation to study these two versions of PMCLP is that cellphone towers can have elliptically shaped coverage area, therefore, to determine what are the best locations to place $m$ cellphone towers to maximize the amount of the population covered by their signal, an elliptical PMCLP is better-suited \cite{canbolat}.

Only two articles have been found published in the literature that study this problem. In \citeonline{canbolat}, a mixed-integer non-linear programming method was proposed as a first approach to the problem. For some instances, the method took too long and did not find an optimal solution. For this reason, a heuristic method was developed using a technique called Simulated Annealing. Solutions for the instances that timed-out with the first method were then obtained. The problem was further explored in \citeonline{andreta} which proposed a deterministic method that showed better performance obtaining optimal solutions for the instances, which the first method could not. Also, in \citeonline{andreta}, the version of the problem where the ellipses could be freely rotated was introduced and an exact method, which could not find optimal solutions for large instances; and a heuristic method were proposed for it. Despite the similarities, none of the works cited above base their development on the maximum covering by disks algorithms found in the literature.

The main results of this work are found in \autoref{chapter:ellipses_n} where a new algorithm for the version of the elliptical PMCLP where the ellipses can be freely rotated is developed. The proposed algorithm has a runtime complexity of $\bigO(n^{3m})$ and one of its most important steps is finding every solution of a not-previously-known problem. In \autoref{chapter:e3p}, this problem is introduced and an algorithm for it is presented. The rest of the work is structured in the following way: \autoref{chapter:definitions} introduces some definitions and results that are used throughout the next chapters; in \autoref{chapter:pmclp}, the maximum covering by disks problem is studied and a $\bigO(n^{2m})$ algorithm is proposed; in \autoref{chapter:ellipses}, the maximum covering by ellipses is introduced and the algorithm for the disks case is adapted for it; finally, \autoref{chapter:future_work} presents a conclusion as well as what is left as future work. Also, \autoref{chapter:ellipses_intersection} determines with detail the intersection of two ellipses, which is used in the algorithm developed in \autoref{chapter:ellipses}.

