The Minimum Cover Problem --also known as just the Set Cover Problem--, and the Maximal Covering Problem are the two main types of optimal covering problems found in the literature \cite{karatas}. 

One of the 21 Karp's NP-Complete problems\footnote{The decision version, which asks if there is a cover of size $k$, is NP-Complete.} \cite{karp}, the Minimum Cover Problem is very well explored and considered to be a classic. 
Given a demand set along with a collection of subsets of the demand set, the problem is to determine the minimum number of elements from the collection of subsets needed to cover the whole demand set. One of its most famous examples is the Minimum Vertex Cover defined over graphs, where the vertex set has to be covered by a subset of edges.

The second type of covering problems arose from the fact that covering almost all the demand set can be a lot cheaper than having to cover it all \cite{garcia}. This second type is known as \sigla{\MCLP}{Maximal Covering Location Problem} and was introduced in \citeonline{church:1974}.
In this first study, the author defined the problem on graphs, and the objective was to maximize the coverage of a demand set, which was a subset of the graph's vertices, by choosing the location of a facility set, which covered any vertex within a given coverage radius.

Just like the Minimum Cover problem, MCLP is NP-Hard \cite{hatta:2013} and both deterministic, using integer programming, and heuristic methods have been proposed to solve it. A very complete survey of developments in this area can be found in \citeonline{mclp_review}.

In \citeonline{church:1984}, a new kind of MCLP named \sigla{\PMCLP}{Planar Maximal Covering Location Problem} was introduced. Unlike its predecessor, this version of the problem was not defined on graphs. Instead, on PMCLP, the demand set and the facilities are located in $\R^2$, and a facility's coverage area is determined by a distance function.
Initially, the Euclidean distance was considered and the idea behind the method proposed in \citeonline{church:1984} was to convert an instance of PMCLP into an instance of MCLP, and then utilize any of the previous developed exact methods to obtain a solution for PMCLP. This reduction was done by identifying a \sigla{CLS}{Candidate Locations Set}, which represented the possible locations that needed to be evaluated for every facility, such that the optimal solution could be found. From the CLS, a network was built on which MCLP could be applied. Generating the CLS specifically for the case of Euclidean distance will be described here on \autoref{chapter:pmclp}.

Furthermore, some variations of PMCLP can also be found in the literature: in \citeonline{younies}, PMCLP was studied under the block norm distance; in \citeonline{mumtaz}, a mean-shift algorithm for large scale\footnote{Numerical experiments were done for up to $3000$ points.} PMCLP was proposed; and in \citeonline{bansal} a version with partial coverage and rectangular demand and facility zones was introduced.

PMCLP under Euclidean distance is also found in the literature as the \sigla{MCD}{Maximum Covering by Disks} problem.
Early works only tackled the one-disk version of it. In \citeonline{chazelle:1986}, a $\bigO(n^2)$ algorithm, which still stands as the best in terms of run-time complexity, was proposed beating the prior $\bigO(n^2\log{n})$ algorithm created by \citeonline{drezner}.
The $m$ disks version of MCD was studied in \citeonline{cabello:2006}, which had as its most important result a $(1-\epsilon)$-approximation algorithm which runs in $\bigO(n\log{n})$. To achieve its main goal, however, they developed a deterministic $\bigO(n^{2m-1}\log{n})$ algorithm which gets employed into their approximation scheme.
Additionally, in \citeonline{aronov:2008}, one-disk maximum covering is proven to be 3SUM-HARD. This means that maximizing the number of points covered by a disk is as hard as finding three real numbers that sum to zero among $n$ given real numbers.

We study two versions of PMCLP with elliptical coverage facilities in this work. For both of them, each ellipse is defined to have a fixed shape and an undefined location, which is part of the solution.
In the first version, introduced in \citeonline{canbolat}, all the ellipses are restricted to be axis-parallel, while in the second version, introduced in \citeonline{andreta}, this constraint is dropped, and all the ellipses can be freely rotated.
The first version will be referred to as \sigla{MCE}{Maximum Cover by Ellipses} and the second one as  \sigla{MCER}{Maximum Covering by Ellipses with Rotation}.

The main practical motivation to study these two versions of PMCLP is that cellphone towers can have an elliptically shaped coverage area. Then, to determine what are the best locations to place $m$ cellphone towers to maximize the amount of the population covered by their signal, an elliptical PMCLP is better-suited \cite{canbolat}.

It is fair to say that PMCLP with elliptical coverage has not been vastly studied as only two articles have been found on it. In \citeonline{canbolat}, a mixed-integer non-linear programming method was proposed as a first approach to the problem. For some instances, the method took too long and did not find an optimal solution. Because of that, a heuristic method was developed using a technique called Simulated Annealing, which was able to obtain solutions for the instances proposed in that study.
The problem was further explored in \citeonline{andreta}, which introduced the version where the ellipses can be freely rotated, to which an exact and a heuristic method was proposed, and developed a new method for the axis-parallel version of the problem, which was able to obtain optimal solutions for instances that the method proposed by \citeonline{canbolat} could not.
The exact method for the version with rotation could not obtain optimal solutions within a predefined time limit for several instances, the heuristic method though, returned solutions for every instance, and impressively enough, obtained optimal solutions for every verifiable instance.


The main results of this work are presented in \autoref{chapter:e3p} and  \autoref{chapter:mcer}. 
In \autoref{chapter:e3p}, we introduce a new geometry problem and propose an algorithm for it.
In \autoref{chapter:mcer}, we use the algorithm developed in \autoref{chapter:e3p} to proposed a new algorithm for MCER. This new algorithm is proved to have a runtime complexity of $\bigO(mn^{3m+1})$, and in \autoref{chapter:numerical}, we analyze several numerical experiments to show its effectiveness and compare it to methods of previous works.
The rest of our work is structured in the following way: In \autoref{chapter:definitions}, some definitions and results that are used throughout the next chapters are introduced; in \autoref{chapter:pmclp}, the maximum covering by disks problem is studied, and an algorithm is proposed for it; in \autoref{chapter:mce}, the maximum covering by ellipses is introduced, and the algorithm for the disks case is adapted for it; in \autoref{chapter:numerical}, numerical experiments are analyzed, and implementation details are given; finally, a conclusion is presented \autoref{chapter:future_work}, along with some suggestions of what can be done in future works on this subject.