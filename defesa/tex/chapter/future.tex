In this work, we studied two problems of maximum planar covering using ellipses, with most of the substantial results being for the less previously-studied problem where the ellipses can be freely rotated, which we referred to as MCER. We devoted \autoref{chapter:e3p} and \autoref{chapter:mcer} of our work for the development of an exact algorithm for this problem, which depended on the development of an algorithm for a never-studied-before geometric subproblem, which we referred to as E3P. 

An exact algorithm was also developed for the other problem studied in this work, which the ellipse could not be rotated. We created an algorithm for it in \autoref{chapter:mce} based on an algorithm for maximum covering using disks, which we introduced in \autoref{chapter:pmclp}.

In \autoref{chapter:numerical}, numerical experiments were run for both algorithms. We first used instances found in the literature, and then proposed new instances to further analyze the performance of our algorithms.
Even though the exponential nature of both algorithms proposed by us, 
in \autoref{chapter:numerical}, we gave several improvements suggestions, which allowed our implementations to obtain optimal solutions for every previously published instance, including instances that no optimal solutions were obtained before, plus some fairly large new ones.


We believe there is plenty of room for furtherly working on these problems.
Back in \autoref{chapter:e3p}, we raised the attention for some possible properties which could be used in the development of an improved version of our algorithm for E3P.
In \autoref{chapter:numerical}, we observed that the bounds for the algorithms proposed by us might be a little loose, obtaining tighter bounds might be possible.
Some other suggestions for future work are: modeling the problems as Linear Integer Programming problems, which could be an alternative to backtracking the optimal solution, done in the algorithms for both problems; adapting the  approximation algorithm for the maximum covering by disks problem developed in \citeonline{cabello:2006} for the ellipses case; and developing and analyzing heuristics which avoid the backtracking phase of both algorithms.