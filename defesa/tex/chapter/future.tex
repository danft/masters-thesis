To take advantage of the great amount of work found in the literature,
 we decided to first introduce the planar maximal covering by disks problem, develop a method for it, and just then adapt it for the ellipses case. It turned out that because of the similarities between the two problems, adapting was possible and actually very simple. This made the method developed by us have a very different approach than the ones in \cite{andreta} and \cite{canbolat}. The next step is to implement it and compare the results that \cite{andreta} obtained.
 
For the next step of our master's research we set the following objectives as primary:

\begin{itemize}
    \item Study the $(1-\epsilon)$-approximation method for the planar covering with disks in \cite{cabello:2006} and develop an adapted version of the algorithm for ellipses with the same time complexity of $\bigO(n\log{n})$.
    
    \item Develop an exact method for the version of the problem introduced in \cite{andreta} where the ellipses can be freely rotated.
    
\end{itemize}

The following goals are set as secondary:

\begin{itemize}
    \item Develop a probabilistic approximation algorithm based on \cite{aronov:2008} which proposed a Monte Carlo approximation for the problem of finding the deepest point in a arrangement of regions. The method runs in $\bigO(n\epsilon^2\log{n})$ and can be applied to solve the case with one ellipse. The case with more than one ellipse is left as a challenge for us for the next steps of our research.
    
    \item In \cite{zhou}, the task of finding every center candidate, after eliminating all the non-essential ones, is done in $\bigO(n^5)$ run-time complexity. We want to generalize this for the elliptical distance function and achieve a better run-time complexity. We also intend to use the mean-shift algorithm to try to develop a greedy version for the ellipses version.
\end{itemize}

%Also, the version of the problem where every ellipse can be freely rotated is set as a primary goal for this master's research. In \cite{andreta}, the deterministic method developed by them could not obtain solutions for moderate-to-large instances within reasonable time. Because of that, a stochastic global optimization method was proposed, it performed well and was able to obtain an optimal solution for small cases. The goal we have in mind for the future is to develop an exact method that takes into consideration the algorithm we developed for the axis-parallel version of the problem and compare the results.

%Finally, as a secondary goal, we want to 

