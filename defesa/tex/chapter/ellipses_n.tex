This section introduces the covering problem that does not have any axis-parallel constraint for the ellipses, we denote this version of the problem as $MCER$. The removal of this constraint introduces a new variable which is responsible for determining the rotation angle of every ellipse.

An instance of the non-axis-parallel is defined exactly like the axis-parallel one on \autoref{chapter:ellipses}. It is given by a set of demand points $\Pp=\{p_1, \dots, p_n\}$, with every point having a unitary weight, and a set of ellipses $\E=\{E_1, \dots, E_m\}$, with fixed shape parameters $(a_i, b_i) \in \R^2_{>0}$, $i \in \{1, \dots, n\}$ with the additional condition that $a_i > b_i$. Given an instance of $MCER$, we define $Q:=(q_1, \dots, q_m) \in \R^{2m}$ to be the centers of each ellipse, $\Theta:=(\theta_1, \dots, \theta_m) \in [0, 2\pi]^m$ to be the angle of rotation of each ellipse and $E_i(q_i, \theta_i)$ to be the coverage region of ellipse $E_i$ with its center at point $q_i$ rotated by angle $\theta_i$. Also, we denote $\Pp \cap E_i(q_i, \theta_i)$ as the set of points covered by $E_i$ on this configuration. Therefore $MCER$ is defined as the problem of determining $Q$ and $\Theta$ (placing and rotating each ellipse) to maximize the number of points covered by the $m$ ellipses, which is given by \autoref{eq:optMCEn}.

\begin{equation}\label{eq:optMCEn}
\max_{Q, \Theta}{\left|\bigcup_{i=1}^{m} \Pp \cap E_i(q_i, \theta_i)\right|}.
\end{equation}

An additional notation is used on this chapter, we define $\tilde{E_i}(q_i, \theta_i)$ to be the set of points on the border of $E_i(q_i, \theta_i)$, we will specially use the operation $\Pp \cap \tilde{E_i}(q_i, \theta_i)$ referring to the set of points from $\Pp$ that lie on the border of $E_i(q_i, \theta_i)$.


\begin{lema}\label{lema:mce_2b}
	Let $(\Pp, \E)$ be an instance of $MCER$. In an optimal solution of $MCER$, for any $E_j \in \E$, such that $|\Pp \cap E_j(q_j, \theta_j)|\ge2$, there is $q_j'$ such that $\Pp \cap E_j(q_j', \theta_j)=\Pp \cap E_j(q_j, \theta_j)$ and $\Pp \cap \tilde{E_j}(q_j', \theta_j) \ge 2$.
\end{lema}

\begin{demonstracao}
	First, the angle of rotation can be ignored as it does not change.
	
	Let $A=\Pp \cap E_j(q_j, \theta_j)$ be the set of points covered by $E_j$ and $X=\cap_{p \in A}E_j(p, \theta_j)$ be the region of intersection of ellipses centered at each point from $A$.
	
	As it was shown on \autoref{chapter:ellipses}, $X$ is a region that is limited by arcs of ellipses. As this region is the non-empty intersection of more than one ellipse, there are at least two of these arcs that encounter at one point creating a vertice. Selecting any of these vertices as $q_j'$ will make $|\Pp \cap E_j(q_j', \theta_j)| \ge 2$.
	
	
\end{demonstracao}

\begin{definicao}\label{def:feasible_angle}
	Let $E$ be an ellipse and $u, v \in \R^2$. An angle $\theta \in [0, \pi]$ is said to be $(E, u, v)$-feasible if there is $q \in \R^2$ such that $\{u, v\} \subset \tilde{E}(q, \theta)$.
\end{definicao}

\begin{lema}\label{lema:3pnts}
	Let $(\Pp, \E)$ be an instance of $MCER$, in an optimal solution, for any $E_j \in \E$, such that $|\Pp \cap E_j(q_j, \theta_j)|>2$, one of the two cases is true:
	
	\begin{enumerate}
		\item There is $q', \theta'$, and $\{u, v, w\} \subset \Pp \cap E_j(q_j, \theta_j)$, such that $\{u, v, w\} \subset \tilde{E}(q', \theta')$.
		
		\item Let $A=\Pp \cap E_j(q_j, \theta_j)$, and $u, v \in A$ such that there exists $\hat{q}_j$ such that $\{u, v\} \subset \tilde{E_j}(\hat{q}_j, \theta_j)$ and $A \subset E_j(\hat{q}_j, \theta_j)$. Then for any $(E_j, u, v)$-feasible angle $\theta \in [0, 2\pi]$, there exists $\bar{q}_j$ such that $\{u, v\} \subset \tilde{E_j}(\bar{q}_j, \theta)$ and $A \subset E_j(\bar{q}_j, \theta)$.
	\end{enumerate}
\end{lema}

The first case of \autoref{lema:3pnts} is saying that there is another optimal solution which has $E_j$ covering the same set of points, but with three points on its border. 

The second case says that after fixing a pair of points on the border of $E_j$ maintaining the covered set, for any angle that allows the two points to stay on the border of $E_j$ the covered set will stay the same.

The idea to prove \autoref{lema:3pnts} is that after fixing $u, v$ on the border of $E_j$, which is possible by \autoref{lema:mce_2b}, the movement of rotation and translation while keeping $u, v$ on the border is continuous. Because of that, the negation of case two implies case one and vice versa.


As a consequence of \autoref{lema:3pnts}, from any optimal solution of $MCER$ it is possible to obtain another optimal solution which every ellipse has one, two or three points on its border.
This result cuts the search space to $\bigO(mn^3)$ which is a very nice gain from where we started, with a infinite search space. Breaking down every possible case, for every ellipse in an optimal solution we obtain:

\begin{itemize}
	\item The ellipse covers only one point. You could place the ellipse so it has one point on its border, satisfying the claim above.
	\item The ellipse covers exactly two points. Using \autoref{lema:mce_2b}, there is a solution with those two points on the border.
	\item The ellipse covers more than two points. First we can use \autoref{lema:mce_2b} to fix two points on the border. By \autoref{lema_3} the angle can be ignored as long as the two fixed points are on the border or exists another optimal solution with three (or more) points on the border.
\end{itemize}

From those cases, three sub-problems need to be solved so we can have a method for $MCER$.

The first one is covering one point with an ellipse which is trivial. For a point $(x, y) \in \R$ just set the center of the ellipse to be $(x, y) \in \R$.

The second problem and third problems deserve sections on their own.

\section{Two points on the border}

Because of the case 2 of \autoref{lema:3pnts}, we want to find for every pair of points the set of points an ellipse with them on its border can cover


 