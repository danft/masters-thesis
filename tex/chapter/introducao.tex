There are two main types of optimal covering problems found in the literature: the minimum cover problem, also known as just set cover, and the maximal covering problem \cite{karatas}. 

One of the 21 Karp's NP-Hard problems \cite{karp}, the minimum cover is considered to be a classic and very well studied problem. On it, a demand set has to be covered, and the cost of doing it has to be minimized. One of its most famous example is the minimum vertex cover defined over graphs where the vertex set has to be covered by a subset of edges.

The second type of covering problems arose from the fact that covering almost all the demand set can be a lot cheaper than having to cover it all \cite{garcia}. This second type is known as \sigla{\MCLP}{Maximal Covering Location Problem} and was introduced in \cite{church:1974}. In this first study, it is defined on a network with demand nodes, a facility set is also given and a solution maximizes the demand coverage satisfying the constraint that only a subset of the facilities are used. Just like the minimum cover, \MCLP is a NP-Hard problem \cite{hatta:2013} and both deterministic, using integer programming \cite{church:1974}, and heuristic methods \cite{revelle:2008} have been proposed for it. 

In \cite{church:1984} a new kind of MCLP named \sigla{\PMCLP}{Planar Maximal Covering Location Problem} was introduced. This version of the problem was not defined on a network, instead the demand set and the facilities are located in $\R^2$, having the coverage area of a facility be defined by a distance function. PMCLP is said to have been studied under euclidean and rectilinear distance functions \cite{younies}. The euclidean norm PMCLP, which has a lot of results that are applied for the elliptical PMCLP, is also found in the literature as the problem of maximization of points covered by a fixed number of unit disks \cite{cabello:2006}. The version where only one disk is used is studied in \cite{chazelle:1986}, a $\bigO(n^2)$ algorithm is proposed, improving the $\bigO(n^2\log{n})$ algorithm that was created in \cite{drezner}. Furthermore, in \cite{aronov:2008} a lower bound result is stabilised, it proves that one disk maximal covering is a 3SUM-HARD problem, which means that it is as difficult as finding three real numbers that sum to zero among $n$ given real numbers. 
The version with $m$ unit disks was studied in \cite{cabello:2006} which developed a $\bigO(n^{2m-1}\log{n})$ deterministic algorithm and a $(1-\epsilon)$-approximation method, for any $\epsilon > 0$, that runs in $\bigO(n\log{n})$.


The difference between the regular PMCLP and the one studied in this work is that the shape of the coverage area of a facility is determined by an ellipse. The main motivation to study this modified version is that cellphone towers can have elliptical shaped coverage area, so in order to determine what are the best locations to place $m$ cellphone towers to maximize the amount of the population covered by its signal, an elliptical PMCLP is better suited \cite{canbolat}. Only two articles have been found published in the literature that study this problem. In \cite{canbolat}, a mixed non-linear programming method was proposed as a first approach to the problem. For some instances the method took too long and did not find the optimal solution. For this reason a heuristic method was developed using a technique called Simulated Annealing, solutions for the instances that timed-out with the first method were then obtained. The problem was further explored in \cite{andreta} which proposed a deterministic method that showed better performance obtaining the optimal solutions for the instances which the first method could not. Also, in \cite{andreta}, a version of the problem where every ellipse can be freely rotated was introduced and an exact method, which could not find the optimal solutions for large instances, and a heuristic method were proposed for it.

This work is structured in the following way: \autoref{chapter:definitions} introduces some definitions and results that are used throughout the next chapters; in \autoref{chapter:pmclp}, the maximal covering by disks problem is studied and a $\bigO(n^{2m})$ is proposed; in \autoref{chapter:ellipses}, the maximal covering by ellipses is introduced and the algorithm for the disks case is adapted for it; finally, \autoref{chapter:future_work} presents what is left as future work. Also, the \autoref{chapter:ellipses_intersection} determines with detail the intersection of two ellipses, which is used in the algorithm developed in \autoref{chapter:ellipses}.

