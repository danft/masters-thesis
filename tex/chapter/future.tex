In this work, we defined the problem of planar maximal covering by disks, we called it $MCD$, and proposed a method for it based on previous works that we found. This method established the basis of our work, as after defining the ellipses counterpart problem, we developed a method based on that algorithm which needed very little modification. To compare results with \cite{andreta}, a small change needs to be done in the algorithm because their version of the problem contains another constraint: out of the $m$ ellipses given, the solution can use exactly $k$ of them. This change as well as the comparison of results is left as future work.

The $(1-\epsilon)$-approximation method for the planar covering with disks in \cite{cabello:2006} is also a subject of future work and is set as the primary goal for this master's research. We want to develop an adapted version of the algorithm for ellipses with the same time complexity of $\bigO(n\log{n})$.

Also, in \cite{aronov:2008} a Monte Carlo approximation is presented for the problem of finding the deepest point in a arrangement of regions. The method runs in $\bigO(n\epsilon^2\log{n})$ and can be applied to solve the case with one ellipse. An adaptation for the multiple ellipses case is left as a secondary goal for the final version of this work.

Finally, the version of the problem where every ellipse can be freely rotated is left as a subject of study for the future. In \cite{andreta}, the deterministic method developed by them could not obtain solutions for moderate-to-large instances within reasonable time. Because of that, a stochastic global optimization method was proposed, it performed well and was able to obtain an optimal solution for small cases. The goal we have in mind for the future is to develop an exact method that takes into consideration the algorithm we developed for the axis-parallel version of the problem and compare the results.