In this appendix the intersection of two ellipses with the same shape parameters $(a,b) \in \R^2_{>0}$ is described with more detail, as well as determining the functions $\Gamma_+(i,j)$ and $\Gamma_-(i,j)$ for two ellipses that intersect.

\section{Intersection}

Let $E_1$ and $E_2$ be two ellipses that the intersection will be determined here. Without loss of generality, let us assume that $E_1$ is at the origin and $E_2$ is located at the center $(h,k) \in \R^2$. Their equations are given by

\begin{align*}
    \frac{x^2}{a^2} + \frac{y^2}{b^2} = 1 && (E_1)\\
    \frac{(x-h)^2}{a^2} + \frac{(y-k)^2}{b^2} = 1 && (E_2)
\end{align*}

As they are both equal to one we can get the following

\begin{align*}
    b^2x^2 + a^2y^2 = b^2(x-h)^2 + a^2(y-k)^2 \\
    b^2(-2xh + h^2) + a^2(-2yk + k^2) = 0\\
    x(2hb^2) = b^2h^2 + a^2(-2yk + k^2)\\
    x = y\frac{-2yka^2}{2hb^2} + \frac{b^2h^2 + a^2k^2}{2hb^2}
\end{align*}

Which can be rewritten as

\begin{align*}
    x = y\alpha + \beta
\end{align*}

with the constants $\alpha$ and $\beta$ being

\begin{align*}
    \alpha = \frac{-2yka^2}{2hb^2} \\
    \beta = \frac{b^2h^2 + a^2k^2}{2hb^2}
\end{align*}

Then replacing it back to the equation of $E_1$ we get

\begin{align*}
    \frac{(y\alpha + \beta)^2}{a^2} + \frac{y^2}{b^2} = 1\\
    b^2(y\alpha + \beta)^2 + y^2a^2 - a^2b^2 = 0\\
    y^2(b^2\alpha^2 + a^2) + y(2\beta\alpha b^2) + b^2\beta^2 -a^2b^2 = 0
\end{align*}

Which is a second degree polynomial, therefore, $E_1$ and $E_2$ intersect if, and only if the roots of the polynomial are real. The intersection points itself can be obtained by solving the polynomial for $y$ and applying its value onto the $x=y\alpha + \beta$ equation.

\subsection{Determining $\Gamma_+(i,j)$ and $\Gamma_-(i,j)$}

Let us assume that $E_1$ and $E_2$, each one with shape parameters $(a,b) \in \R^2_{>0}$, intersect at $p_1$ and $p_2$. Then, to determine $\Gamma_+(1,2)$ and $\Gamma_-(1,2)$, we need to first determine the angles of intersection of $p_1$ and $p_2$ on $E_1$. For that, we will use the curve defined in \autoref{eq:parametric_ellipse} because it is easier to work with angles here.

Given a point $(x,y)$, to find the angle it makes with the major axis, from \autoref{eq:parametric_ellipse}, we can get that

\begin{align*}
\dfrac{y-q_y}{x-q_x} = \dfrac{b}{a}\tan{t}\\
t=\arctan\left(\dfrac{a}{b} \dfrac{y-q_y}{x-q_x}\right)
\end{align*}

As the image of $\arctan$ is $[-\frac{\pi}{2}, \frac{\pi}{2}]$, we need to check the sign of $x-q_x$ to determine the angle in $[0, 2\pi]$. After that, we can get the two angles that represent the intersection points $p_1$ and $p_2$ on $E_1$.

To find out which one of the angles are $\Gamma_+(1,2)$, we need to go further and determine the derivative of $\gamma(t)$ which is going to be used to determine the vectors tangent to the ellipses at the intersection points.

\begin{equation}\label{eq:der_parametric_ellipse}
\gamma'(t) = \left\{
\begin{array}{l}
x'(t)= -a\sin{t}\\
y'(t)=b\cos{t}
\end{array}
\right.
\end{equation}

Let $\gamma_1$ and $\gamma_2$ be the curves describing $E_1$ and $E_2$ respectively. Also, let $s_1$ be the angle, such that $\gamma_1(s_1)=p_1$, and $t_1$ be the angle, such that $\gamma_2(t_1)=p_1$.
Then, the tangent vectors to the $E_1$ and $E_2$ at $p_1$ are $\gamma_1'(s_1)$ and $\gamma_2'(t_1)$ respectively.

\begin{figure}[H]
\centering

    \caption{Determining $\Gamma_+(1,2)$}
    

%\tikzset{every picture/.style={line width=0.75pt}} %set default line width to 0.75pt        

\begin{tikzpicture}[x=0.75pt,y=0.75pt,yscale=-1,xscale=1]
every edge quotes/.append style = {anchor=south, sloped}

%uncomment if require: \path (0,191); %set diagram left start at 0, and has height of 191

%Shape: Ellipse [id:dp10634347053631776] 
\draw   (68.5,97.75) .. controls (68.5,67.24) and (115.62,42.5) .. (173.75,42.5) .. controls (231.88,42.5) and (279,67.24) .. (279,97.75) .. controls (279,128.26) and (231.88,153) .. (173.75,153) .. controls (115.62,153) and (68.5,128.26) .. (68.5,97.75) -- cycle ;
%Shape: Ellipse [id:dp7816950476685691] 
\draw  [color={rgb, 255:red, 163; green, 163; blue, 163 }  ,draw opacity=1 ] (167.5,103.75) .. controls (167.5,73.24) and (214.62,48.5) .. (272.75,48.5) .. controls (330.88,48.5) and (378,73.24) .. (378,103.75) .. controls (378,134.26) and (330.88,159) .. (272.75,159) .. controls (214.62,159) and (167.5,134.26) .. (167.5,103.75) -- cycle ;
%Straight Lines [id:da36505632658461096] 
\draw    (234,52) -- (185.87,33.72) ;
\draw [shift={(184,33)}, rotate = 381.03999999999996] [color={rgb, 255:red, 0; green, 0; blue, 0 }  ][line width=0.75]    (10.93,-3.29) .. controls (6.95,-1.4) and (3.31,-0.3) .. (0,0) .. controls (3.31,0.3) and (6.95,1.4) .. (10.93,3.29)   ;

%Straight Lines [id:da357569598162492] 
\draw    (234,52) -- (286.05,39.95) ;
\draw [shift={(288,39.5)}, rotate = 526.97] [color={rgb, 255:red, 0; green, 0; blue, 0 }  ][line width=0.75]    (10.93,-3.29) .. controls (6.95,-1.4) and (3.31,-0.3) .. (0,0) .. controls (3.31,0.3) and (6.95,1.4) .. (10.93,3.29)   ;

%Straight Lines [id:da9574900886541131] 
\draw    (212,149) -- (173.91,137.34) ;
\draw [shift={(172,136.75)}, rotate = 377.03] [color={rgb, 255:red, 0; green, 0; blue, 0 }  ][line width=0.75]    (10.93,-3.29) .. controls (6.95,-1.4) and (3.31,-0.3) .. (0,0) .. controls (3.31,0.3) and (6.95,1.4) .. (10.93,3.29)   ;

%Straight Lines [id:da5801314237314061] 
\draw    (212,149) -- (261.53,140.1) ;
\draw [shift={(263.5,139.75)}, rotate = 529.8199999999999] [color={rgb, 255:red, 0; green, 0; blue, 0 }  ][line width=0.75]    (10.93,-3.29) .. controls (6.95,-1.4) and (3.31,-0.3) .. (0,0) .. controls (3.31,0.3) and (6.95,1.4) .. (10.93,3.29)   ;


% Text Node
\draw (216,26) node [scale=0.7]  {$\gamma _{1} '( s_{2})$};
% Text Node
\draw (312.5,36.5) node [scale=0.7]  {$-\gamma _{2} '( t_{2})$};
% Text Node
\draw (284,143) node [scale=0.7]  {$\gamma _{1} '( s_{1})$};
% Text Node
\draw (147,134) node [scale=0.7]  {$-\gamma _{2} '( t_{1})$};
% Text Node
\draw (214,163) node [scale=0.7]  {$p_{1}$};
% Text Node
\draw (237,41) node [scale=0.7]  {$p_{2}$};
% Text Node
\draw (56,93) node   {$E_{1}$};
% Text Node
\draw (390,97) node   {$E_{2}$};


\coordinate (O) at (212,149) ;
\coordinate (A) at (261.53,140.1) ;
\coordinate (B) at (173.91,137.34) ;

\coordinate (O2) at (234,52);
\coordinate (A2) at (185.87,33.72);
\coordinate (B2) at (286.05,39.95);

\draw (214,130) node [scale=0.7]  {$\theta$};

\draw pic[draw=black, -, angle eccentricity=1.2, angle radius=0.2cm]
{angle=A--O--B};

\draw pic[draw=black, -, angle eccentricity=1.2, angle radius=0.2cm]
{angle=A2--O2--B2};

\end{tikzpicture}

    \fautor
    \label{fig:ellipse_gamma}
\end{figure}

The following lemma states a relation between $s_1$ and $\Gamma_+(1,2)$

\begin{lema}\label{lema:lema_ellipse_angle}
	Let $\theta$ be the angle between $\gamma_1'(s_1)$ and $-\gamma_2'(t_1)$. Then, $\theta \le \pi$ if, and only if $\Gamma_+(1,2)=s_1$.
\end{lema}

Instead of a formal proof of \autoref{lema:lema_ellipse_angle}, a graphical explanation using \autoref{fig:ellipse_gamma} is provided.


First, let us state some facts that can also be seen in \autoref{fig:ellipse_gamma}

\begin{itemize}
    \item $E_1 \cap E_2$ is convex and bounded by two arcs, one from each ellipse.
    \item Starting at any of the intersection points, one of the $E_1 \cap E_2$ arcs will be clockwise-oriented and the other, counter-clockwise-oriented. In \autoref{fig:ellipse_gamma}, for example, it is clear that only the $E_1$ arc starting at $p_1$, ending at $p_2$, is counter-clockwise-oriented.  
    
    
    \item The counter-clockwise-oriented arc starting at $\Gamma_+(1,2)$ is from the ellipse $E_1$.
\end{itemize}


Let us assume that $p_1$ is the intersection point which is the opening angle $\Gamma_+(1,2)$. Then, the vectors $\gamma_1'(s_1)$ and $-\gamma_2'(t_1)$ are tangent to the $E_1 \cap E_2$ area at point $p_1$. Because of the convexity of $E_1 \cap E_2$, the angle between $\gamma_1'(s_1)$ and $-\gamma_2'(t_1)$ has to be less than or equal to $\pi$ (see \autoref{fig:ellipse_gamma}), which is what \autoref{lema:lema_ellipse_angle} says. It is easy to prove the converse by proving the contra-positive assuming that $p_1$ is the point which determines the angle $\Gamma_-(1,2)$.

Lastly, in \autoref{fig:ellipse_gamma}, it can be seen that if one the intersection points is classified as $\Gamma_+(1,2)$ the other will necessarily be classified as $\Gamma_-(1,2)$. This gives us all we need to implement \autoref{algoritmo:mce}.

%It is worth noting that computationally, this classification can be done taking the cross product of the two vectors and checking its signal, negative cross products imply an angle greater than $\pi$. Also, determining the polar angle of a point in $\R^2$ can be done using the $atan2(x,y)$ function which is present in most of the math libraries of modern programming languages.

