
		%\title{Answer to Reviewer #1}
		%\maketitle
		
		\begin{letter}{}

		\opening{Answer to Reviewer \#2}
		%	\opening{Dear Prof. Roman Slowinski\\Coordinating Editor of EJOR}
		
		First, thanks for dedicating your time to revise our work. We found every one of the points made of great importance for us to achieve a top quality paper.
		
		The changes we have made to the text are highlighted by different colors. 
		\begin{itemize}
			\item The {\color{blue} blue} highlights are small corrections that were pointed out by the board that revised the dissertation from which this paper originated.
			
			\item The {\color{Red} red} highlights are corrections concerning the points made by reviewer \#1.
			
			\item The {\color{Green} green} highlights are corrections concerning the points made by reviewer \#2.
			
			\item The {\color{Orange} orange} highlights are corrections concerning points made by both reviewers.
		\end{itemize}
		 
		Following we provide answers to each one of your comments:
		
		\textbf{Comment \#1:} `` The highlights (contributions of the paper) and organization of the remaining sections should be presented at the end of Section 1."
		
		We address this issued by adding a paragraph to the end of our introduction, please see page 3 and 4.
		
		\textbf{Comment \#2:} ``In my opinion "fixed shape" ellipses should be replaced by "fixed shape parameters" ellipses, or ellipses with the given semi-major and semi-minor axes. Alternately the term "fixed shape" ellipses should be explained."
		
		We agree that using ``fixed shape" without clearly defining what it is can be confusing to the reader. To address this issue we rewrote some pieces of text were this was being used:
		\begin{itemize}
			\item In the first paragraph of the abstract (page 2), we used the term ``given their major and minor axis" to better convey the same idea.
			
			\item In the ``Introduction" section (page 3), we changed the description of our problem to: ``(...) each ellipse is defined to have fixed major and minor axis and an undefined location (...)."
			
			\item Section 4 had in its title (page 7) ``(...) of An Ellipse Given Its Shape and (...)", we replaced it by ``(...) of An Ellipse Given Its Shape Parameters and (...)."
		\end{itemize}
		
		\textbf{Comment \#3:} ``Please, extend the description of practical applications."
		We address this issued by re-writing a piece of our introduction. The changes are in the ``Introduction" section (page 3). We decided to make a brief comment about the real-world applications of the problem and cite another work that explores this with greater detail. Also, because of the changes made to fit this comment, we ended up rewriting the next paragraph.
		
		\textbf{Comment \#4:} ``The literature review in Introduction is a bit limited and should to be extended analyzing more approaches and references on the subject. It can be expected that beginning from the first publication [8] in 1984 a reasonable number of references on the subject have been published. See, e.g. papers: (...)"
		
		This issued is addressed in the ``Introduction" section on page 2. We agree with you that the literature review needed improvement, this is the sum-up of the changes we made:

		\begin{itemize}
			\item We add a citation to ``H Younies and G O Wesolowsky. Planar maximal covering location problem under block norm distance measure". This shows another example of versions of PMCLP other than the classical one using Euclidean norm.
			\item We add a citation to the survey paper: ``Alan T. Murray. Maximal coverage location problem: Impacts, significance, and evolution.". Instead of giving an extensive review, we use a common practice of citing a survey paper. 
		\end{itemize}
	
		Also, regarding the possible works that we could cite in our paper suggested by you, besides the PHD thesis of Daoqin Tong, who has some of his works cited in the above survey, we think that the other suggested papers seem not to be that closely related to our problem.
		
		\textbf{Comment \#5:} ``Please, clarify what is the critical parameter for your algorithm: the number of ellipses or the number of demand points?"
		
		We addressed this issue in section 8.2 ``New instances" (page 19). Now, we hope it is clear that the critical parameter is the number of ellipses.
		
		\textbf{Comment \#6:} ``In the paper overlapping of covering ellipses is allowed, i.e. a demand point can be covered by more than one ellipse. However, it seems natural that the weight of the demand point covered by various ellipses can differ from the weight of the same point covered by only one ellipse."
		
		We are using the same definition of PMCLP used in the several works we cited, maybe you could elaborate more why this seems natural?
		
		\textbf{Comment \#7:} ``What happens if no overlapping between covering ellipses is allowed. Please, comment on this."
		
		Our algorithm does not support this constraint, and we do not see any trivial modification that could make it do.
		
		Please see: \href{https://github.com/danft/masters-thesis/raw/master/paper/figures/answer-to-reviewer-2.pdf}{https://github.com/danft/masters-thesis/raw/master/paper/figures/answer-to-reviewer-2.pdf} for a counter-example of how our algorithm fails for this constraint. That is, there is a non-overlapping solution, yet none of them is in the set of solutions we look for.
		
		\closing{On behalf of the authors and with our best regards,}
			\end{letter}