In this work, we developed new algorithms for two planar maximum covering by ellipses problems.
After giving improvements suggestions, and implementation details, we analyzed numerical experiments showing the efficiency of both algorithms for the instances of previous works, and for new ones with larger demand and facility sets.

We devoted most of our work to the development of an exact algorithm for MCER.
In the midst of that, a never-studied-before subproblem came up, for which we also proposed an algorithm, which involved determining the roots of a complex polynomial of degree six.
We believe that further exploring this new problem could be the subject of future work as we observed that its solutions, except for one case, always come in pairs; and that this property seems to be directly connected with the complex polynomial, as its roots seem to come in conjugate pairs. These properties, if proved to be true, might be used in the development of a more efficient algorithm.

{\color{Green}
	Finally, we leave some additional suggestions for future works on this subject. We believe that fixing only the area of each ellipse, instead of its shape, can be an interesting extension of MCE and MCER; having only one additional decision variable --the major or minor axis-- might still allow a similar model to be employed, even though extensive work would be necessary to adapt, somehow, the theory developed for this version, especially of MCER.
We also believe that investigating coverage of points by ellipsoids in higher dimensions can be a great subject of future studies taking this work as the starting point; 
coming up with a general theoretical result for a $d$-dimension MCER, allowing it to be transformed into a combinatorial optimization problem like it was done here for the two-dimensional case, seems to be very challenging, and even though there might be no practical applications for this generalization, it is still very interesting from the theoretical perspective. 
%{\color{Red}
%	The last suggestion we give is that 
%	Lastly, as it has been pointed out, the critical parameter of both algorithms developed in this paper is the number of ellipses. We think that trading-off being able to run with large demand sets can be the basis for future work on an alternative algorithm that can obtain optimal solutions for instances with larger sets of ellipses.}
%}